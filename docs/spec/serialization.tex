\section{Serialization}
\label{sec:serialization}

This section defines a binary format, which is used to store \langname
contracts in persistent stores, to transfer them over the wire and to enable
cross-platform interoperation.

Terms of the language described in Section~\ref{sec:language} can be
serialized to array of bytes to be stored in Ergo blockchain (e.g. as
\lst{Box.propositionBytes}).

When the guarding script of an input box of a transaction is validated the
\lst{propositionBytes} array is deserialized to an \langname IR (represented by the
\lst{ErgoTree} class), which can be evaluated as it is specified in
Section~\ref{sec:evaluation}.

Here we specify the serialization procedure in general. The serialization format of
\langname types (\lst{SType} class) and nodes (\lst{Value} class) is specified in
section~\ref{sec:ser:type} and Appendix~\ref{sec:appendix:ergotree_serialization}
correspondingly.

Table~\ref{table:ser:limits} shows size limits which are checked during
contract deserialization, which is important to resist malicious script attacks.

\begin{table}[h]
    \caption{Serialization limits}\vspace{-7pt}
    \label{table:ser:limits}
    \footnotesize
\(\begin{tabularx}{\textwidth}{| l | p{2.5cm} | X |}
    \hline
    \bf{Constant}   & \bf{Value} & \bf{Description} \\
    \hline
    $\MaxVlqSize$  & $10$ & Maximum size of VLQ encoded byte sequence (See VLQ formats~\ref{sec:vlq-encoding})  \\
    \hline
    $\MaxTypeSize$ & $100$ & Maximum size of serialized type term (see Type format~\ref{sec:ser:type}) \\
    \hline
    $\MaxDataSize$ & $4Kb$ & Maximum size of serialized data instance (see Data format~\ref{sec:ser:data}) \\
    \hline
    $\MaxConstSize$ & $=\MaxTypeSize+\MaxDataSize$  & Maximum size of serialized data instance (see Const format~\ref{sec:ser:const}) \\
    \hline
    $\MaxExprSize$ & $4Kb$ & Maximum size of serialized \langname term (see Expr format~\ref{sec:ser:expr}) \\
    \hline
    $\MaxErgoTreeSize$ & $4Kb$ & Maximum size of serialized \langname contract (see ErgoTree format~\ref{sec:ser:ergotree}) \\
    \hline
\end{tabularx}\)

\end{table}

All the serialization formats which are uses and defined thoughout this section are
listed in Table~\ref{table:ser:formats} which introduces a name for each format and also
shows the number of bytes each format may occupy in the byte stream.

\begin{table}[H] \scriptsize
\caption{Serialization formats}\vspace{-7pt}
\label{table:ser:formats}
\(\begin{tabularx}{\textwidth}{| l | l | X |}
    \hline
    \bf{Format} & \bf{\#bytes} & \bf{Description} \\
    \hline
    \lst{Byte} & $1$ & 8-bit signed two's-complement integer \\
    \hline
    \lst{Short} & $2$ & 16-bit signed two's-complement integer (big-endian) \\
    \hline    
    \lst{Int} & $4$ & 32-bit signed two's-complement integer (big-endian) \\
    \hline
    \lst{Long} & $8$ & 64-bit signed two's-complement integer (big-endian) \\
    \hline
    \lst{UByte} & $1$ & 8-bit unsigned integer \\
    \hline
    \lst{UShort} & $2$ & 16-bit unsigned integer (big-endian) \\
    \hline    
    \lst{UInt} & $4$ & 32-bit unsigned integer (big-endian) \\
    \hline
    \lst{ULong} & $8$ & 64-bit unsigned integer (big-endian) \\

    \hline
    \lst{VLQ(UShort)} & $[1..3]$ & Encoded unsigned \lst{Short} value using VLQ. See~\cite{VLQWikipedia,VLQRosetta} and~\ref{sec:vlq-encoding} \\
    \hline    
    \lst{VLQ(UInt)} & $[1..5]$ & Encoded unsigned 32-bit integer using VLQ. \\
    \hline
    \lst{VLQ(ULong)} & $[1..\MaxVlqSize]$ & Encoded unsigned 64-bit integer using VLQ. \\

    \hline
    \lst{Bits} & $[1..\MaxBits]$ & A collection of bits packed in a sequence of bytes. \\
    \hline
    \lst{Bytes} & $[1..\MaxBytes]$ & A sequence of bytes, which size is stored elsewhere or wellknown. \\

    \hline
    \lst{Type} & $[1..\MaxTypeSize]$ & Serialized type terms of \langname. See~\ref{sec:ser:type} \\
    \hline
    \lst{Data} & $[1..\MaxDataSize]$ & Serialized data values of \langname. See~\ref{sec:ser:data} \\
    \hline
    \lst{GroupElement} & $33$ & Serialized elements of eliptic curve group. See~\ref{sec:ser:data:groupelement} \\
    \hline
    \lst{SigmaProp} & $[1..\MaxSigmaProp]$ & Serialized sigma propositions. See~\ref{sec:ser:data:sigmaprop} \\
    \hline
    \lst{AvlTree} & $44$ & Serialized dynamic dictionary digest. See~\ref{sec:ser:data:avltree} \\
    \hline
    \lst{Constant} & $[1..\MaxConstSize]$ & Serialized \langname constants (values with types). See~\ref{sec:ser:const} \\
    \hline
    \lst{Expr} & $[1..\MaxExprSize]$ & Serialized expression terms of \langname. See~\ref{sec:ser:expr} \\
    \hline
    \lst{ErgoTree} & $[1..\MaxErgoTreeSize]$ & Serialized instances of \langname contracts. See~\ref{sec:ser:ergotree} \\
    \hline
\end{tabularx}\)
\end{table}

We use $[1..n]$ notation when serialization may produce from 1 to n bytes depending of
actual data instance.

Serialization format of \ASDag is optimized for compact storage and very fast
deserialization. In many cases serialization procedure is data dependent and thus have
branching logic. To express this complex serialization logic in the specification we
use a \emph{pseudo-language} with operators like \lst{for, match, if, optional}. The
language allows to specify a \emph{structure} out of \emph{simple serialization slots}.
Each \emph{slot} specifies a fragment of serialized stream of bytes, whereas
\emph{operators} specifiy how the slots are combined together to form the resulting
stream of bytes. The notation is summarized in Table~\ref{table:ser:notation}.

\begin{table}[h] \footnotesize
\caption{Serialization Notation}
\label{table:ser:notation}
\(\begin{tabularx}{\textwidth}{| l | X |}
    \hline
    \bf{Notation} & \bf{Description} \\
    \hline
    $\Denot{T}$ where $T$ - type & Denotes a set of values of type $T$  \\
    \hline
    $v \in \Denot{T}$ & The value $v$ belongs to the set $\Denot{T}$ \\
    \hline   
    $v : T$ & Same as $v \in \Denot{T}$ \\
    \hline    
    \lst{match} $(t, v)$ & Pattern match on pair $(t, v)$ where $t, v$ - values \\
    \hline    
    \lst{with} $(Unit, v \in \Denot{Unit})$ & Pattern case \\
    \hline    

    \lst{for}~$i=1$~\lst{to}~$len$ & \multirow{3}{=}{Call 
        the given \lst{serialize} function repeatedly. 
        The outputs bytes of all invocations are concatenated and become 
        the output of the \lst{for} statement. } \\
    ~~\lst{serialize(}$v_i$\lst{)} &  \\
    \lst{end for} & \\
    \hline

    \lst{if}~$condition$~\lst{then} & \multirow[t]{5}{=}{Serialize 
        one of the branches depending of the $condition$.  
        The output bytes of the executed branch becomes the output of the \lst{if} statement. } \\
    ~~\lst{serialize1(}$v_1$\lst{)} &  \\
    \lst{else} & \\
    ~~\lst{serialize2(}$v_2$\lst{)} &  \\
    \lst{end if} & \\
    \hline
\end{tabularx}\)
\end{table}

In the next section we describe how types (like \lst{Int}, \lst{Coll[Byte]},
etc.) are serialized, then we define serialization of typed data. This will
give us a basis to describe serialization of Constant nodes of \ASDag. From
that we will proceed to serialization of arbitrary \ASDag trees.

\subsection{Type Serialization}
\label{sec:ser:type}

For motivation behind this type encoding please see
Appendix~\ref{sec:appendix:motivation:type}.

\subsubsection{Distribution of type codes}
\label{sec:ser:type:codedist}

The whole space of 256 one byte codes is divided as shown in
Figure~\ref{fig:ser:type:codedist}.

\begin{table}[h] \footnotesize
\caption{Distribution of type codes between Data and Function types}\vspace{-7pt}
\label{fig:ser:type:codedist}
\(\begin{tabularx}{\textwidth}{| l | X |}
    \hline
    \bf{Value/Interval} & \bf{Distribution} \\
    \hline
    \lst{0x00} & special value to represent undefined type (\lst{NoType} in \ASDag) \\
    \hline
    \lst{0x01 - 0x6F(111)} & \emph{data types} including primitive types, arrays, options
    aka nullable types, classes (in future), 111 = 255 - 144 different codes \\
    \hline
    \lst{0x70(112) - 0xFF(255)} & \emph{function types} \lst{T1 => T2}, 144 = 12 x 12
    different codes~\footnote{Note that the function types are never serialized in version 1 of the Ergo
    protocol, this encoding is reserved for future development of the protocol.} \\
    \hline 
\end{tabularx}\)

\end{table}

\subsubsection{Encoding of Data Types}

There are eight different values for \emph{embeddable} types and 3 more are reserved
for the future extensions. Each embeddable type has a type code in the range {1,...,11}
as shown in Figure~\ref{fig:ser:type:embeddable}.

\begin{table}[h] \footnotesize
\caption{Embeddable Types}\vspace{-7pt}
\label{fig:ser:type:embeddable}
    \(\begin{tabularx}{\textwidth}{| l | X |}
        \hline
        \bf{Code} & \bf{Type} \\ \hline
1     &   Boolean \\  \hline
2     &   Byte\\  \hline
3     &   Short (16 bit)\\  \hline
4     &   Int (32 bit)\\  \hline
5     &   Long (64 bit)\\  \hline
6     &   BigInt (represented by java.math.BigInteger)\\  \hline
7     &   GroupElement (represented by org.bouncycastle.math.ec.ECPoint)\\  \hline
8     &   SigmaProp \\  \hline
9     &   reserved for Char \\  \hline
10    &   reserved \\  \hline
11    &   reserved \\ \hline 
\end{tabularx}\)
\end{table}

\begin{table}[H] \scriptsize
    \caption{Code Ranges of Data Types}\vspace{-7pt}
    \label{fig:ser:type:datatypes}
    \(\begin{tabularx}{\textwidth}{| l | l | X |}
    \hline
    \bf{Interval}       & \bf{Constructor} & \bf{Description} \\ \hline
    0x01 - 0x0B(11)     &                       & embeddable types (including 3 reserved) \\ \hline
    0x0C(12)            & \lst{Coll[_]}         & Collection of non-embeddable types (\lst{Coll[(Int,Boolean)]}) \\ \hline
    0x0D(13) - 0x17(23) & \lst{Coll[_]}         & Collection of embeddable types (\lst{Coll[Byte]}, \lst{Coll[Int]}, etc.) \\ \hline
    0x18(24)            & \lst{Coll[Coll[_]]}   & Nested collection of non-embeddable types (\lst{Coll[Coll[(Int,Boolean)]]}) \\ \hline
    0x19(25) - 0x23(35) & \lst{Coll[Coll[_]]}   & Nested collection of embeddable types (\lst{Coll[Coll[Byte]]}, \lst{Coll[Coll[Int]]}) \\ \hline
    0x24(36)            & \lst{Option[_]}       & Option of non-embeddable type (\lst{Option[(Int, Byte)]}) \\ \hline
    0x25(37) - 0x2F(47) & \lst{Option[_]}       & Option of embeddable type (\lst{Option[Int]}) \\ \hline
    0x30(48)            & \lst{Option[Coll[_]]} & Option of Coll of non-embeddable type (\lst{Option[Coll[(Int, Boolean)]]}) \\ \hline
    0x31(49) - 0x3B(59) & \lst{Option[Coll[_]]} & Option of Coll of embeddable type (\lst{Option[Coll[Int]]}) \\ \hline
    0x3C(60)            & \lst{(_,_)}           & Pair of non-embeddable types (\lst{((Int, Byte), (Boolean,Box))}, etc.) \\ \hline
    0x3D(61) - 0x47(71) & \lst{(_, Int)}        & Pair of types where first is embeddable (\lst{(_, Int)}) \\ \hline
    0x48(72)            & \lst{(_,_,_)}         & Triple of types  \\ \hline
    0x49(73) - 0x53(83) & \lst{(Int, _)}        & Pair of types where second is embeddable (\lst{(Int, _)}) \\ \hline
    0x54(84)            & \lst{(_,_,_,_)}       & Quadruple of types  \\ \hline
    0x55(85) - 0x5F(95) & \lst{(_, _)}          & Symmetric pair of embeddable types (\lst{(Int, Int)}, \lst{(Byte,Byte)}, etc.) \\ \hline
    0x60(96)            & \lst{(_,...,_)}       & \lst{Tuple} type with more than 4 items \lst{(Int, Byte, Box, Boolean, Int)} \\ \hline
    0x61(97)            & \lst{Any}             & Any type  \\ \hline
    0x62(98)            & \lst{Unit}            & Unit type \\ \hline
    0x63(99)            & \lst{Box}             & Box type  \\ \hline
    0x64(100)           & \lst{AvlTree}         & AvlTree type  \\ \hline
    0x65(101)           & \lst{Context}         & Context type  \\ \hline
    0x66(102)           &                       & reserved for String  \\ \hline
    0x67(103)           &                       & reserved for TypeVar  \\ \hline
    0x68(104)           & \lst{Header}          & Header type  \\ \hline
    0x69(105)           & \lst{PreHeader}       & PreHeader type  \\ \hline
    0x6A(106)           & \lst{Global}          & Global type  \\ \hline
    0x6B(107)-0x6E(110) &                       & reserved for future use  \\ \hline
    0x6F(111)           &                       & Reserved for future \lst{Class} type (e.g. user-defined types)  \\ \hline
    \end{tabularx}\)
\end{table}

For each type constructor like \lst{Coll} or \lst{Option} we use the encoding schema
defined below. Type constructor has an associated \emph{base code} which is multiple
of~$12$ (e.g.~$12$ for \lst{Coll[_]}, $24$ for \lst{Coll[Coll[_]]} etc.).
The base code can be added to the embeddable type code to produce the code of the constructed
type, for example $12 + 1 = 13$ is a code of \lst{Coll[Byte]}. 
The code of type
constructor (e.g. $12$ in this example) is used when type parameter is non-embeddable
type (e.g. \lst{Coll[(Byte, Int)]}). In this case the code of type
constructor is read first, and then recursive descent is performed to read
bytes of the parameter type (in this case \lst{(Byte, Int)}). This encoding
allows very simple and fast decoding by using \lst{div} and \lst{mod} operations.

Following the above encoding schema the interval of codes for data types is divided as
shown in Table~\ref{fig:ser:type:datatypes}.


\subsubsection{Encoding of Function Types}

We use $12$ different values for both domain and range types of functions. This
gives us $12 * 12 = 144$ function types in total and allows to represent $11 *
11 = 121$ functions over primitive types using just single byte.

Each code $F$ in a range of the function types (i.e $F \in \{112, \dots, 255\}$) can be
represented as $F~=~D * 12 + R + 112$, where $D, R \in \{0,\dots,11\}$ - indices of
domain and range types correspondingly, $112$ - is the first code in an interval of
function types.

If $D = 0$ then the domain type is not embeddable and the recursive descent is
necessary to write/read the domain type.

If $R = 0$ then the range type is not embeddable and the recursive descent is necessary
to write/read the range type.

\subsubsection{Recursive Descent}
\label{sec:ser:type:recursive}

When an argument of a type constructor is not a primitive type we fallback to the
simple encoding schema in which case we emit the separate code for the type constructor
according to the table above and descend recursively to every child node of the type
tree.

We do this descend only for those children whose code cannot be embedded in
the parent code. For example, serialization of \lst{Coll[(Int,Boolean)]}
proceeds as the following:
\begin{enumerate}
\item Emit \lst{0x0C} because the elements type of the collection is not embeddable 
\item Recursively serialize \lst{(Int, Boolean)}
\item Emit \lst{0x41(=0x3D+4)} because the first type of the pair is embeddable and its code is~$4$
\item Recursivley serialize \lst{Boolean}
\item Emit \lst{0x02} - the code for embeddable type \lst{Boolean}
\end{enumerate}

More examples of type serialization are shown in Table~\ref{fig:ser:type:examples}
\begin{table}[H] \footnotesize
\caption{Examples of type serialization}\vspace{-7pt}
\label{fig:ser:type:examples}
\(\begin{tabularx}{\textwidth}{| l | c | c | l | c | X |}
\hline
\bf{Type}                &\bf{D} & \bf{R} & \bf{Serialized Bytes} & \bf{\#Bytes} &  \bf{Comments} \\ \hline
\lst{Byte}               &     &     & 2                   &  1     & simple embeddable type     \\ \hline
\lst{Coll[Byte]}         &     &     & 12 + 2 = 14         &  1     & embeddable type in Coll \\ \hline
\lst{Coll[Coll[Byte]]}   &     &     & 24 + 2 = 26         &  1     & embeddable type in nested Coll \\ \hline
\lst{Option[Byte]}       &     &     & 36 + 2 = 38         &  1     & embeddable type in Option    \\ \hline
\lst{Option[Coll[Byte]]} &     &     & 48 + 2 = 50         &  1     & embeddable type in Coll nested in Option   \\ \hline
\lst{(Int,Int)}          &     &     & 84 + 4 = 88         &  1     & symmetric pair of embeddable type   \\ \hline
\lst{Int=>Boolean}       & 4   & 1   & 161 = 4*12+1+112    &  1     & embeddable domain and range  \\ \hline
\lst{(Int,Int)=>Int}     & 0   & 4   & 115=0*12+4+112, 88  &  2     & embeddable range, then symmetric pair    \\ \hline
\lst{(Int,Boolean)}      &     &     & 60 + 4, 1           &  2     & Int embedded in pair, then Boolean     \\ \hline
\lst{(Int,Box)=>Boolean} & 0   & 1   & 0*12+1+112, 60+4, 99 &  3    & func with embedded range, then Int embedded, then Box    \\ \hline
\end{tabularx}\)
\end{table}


\subsection{Data Serialization}
\label{sec:ser:data}

In \langname all runtime data values have an associated type also available
at runtime (this is called \emph{type reification}\cite{Reification}).
However serialization format separates data values from its type descriptors. 
This allows to save space when for example a collection of items is serialized.

It is done is such a way that the contents of a typed data structure can be fully
described by a type tree. For example having a typed data object \lst{d: (Int,
Coll[Byte], Boolean)} we can tell, by examining the structure of the type, that \lst{d}
is a tuple with 3 items, the first item contain 32-bit integer, the second - collection
of bytes, and the third - logical true/false value.

To serialize/deserialize typed data we need to know its \emph{type descriptor} (type
tree). The data serialization procedure is recursive over a type tree and the
corresponding sub-components of the data object. For primitive types (the leaves of the
type tree) the format is fixed. The data values of \langname types are serialized
according to the predefined recursive function shown in Figure~\ref{fig:ser:data} which
uses the notation from Table~\ref{table:ser:notation}.

\begin{figure}[H] \footnotesize
\caption{Data serialization format}\vspace{-7pt}
\label{fig:ser:data}
\(\begin{tabularx}{\textwidth}{| l | l | l | X |}
    \hline
    \bf{Slot} & \bf{Format} & \bf{\#bytes} & \bf{Description} \\
    \hline
    \hline
    \multicolumn{4}{l}{\lst{def serializeData(}$t, v$\lst{)}} \\
    \multicolumn{4}{l}{~~\lst{match} $(t, v)$ } \\

    \multicolumn{4}{l}{~~~~\lst{with} $(Unit, v \in \Denot{Unit})$~~~// nothing serialized } \\
    \multicolumn{4}{l}{~~~~\lst{with} $(Boolean, v \in \Denot{Boolean})$} \\
    \hline
    $~~~~~~v$ & \lst{Byte} & 1 & 0 if $v = false$ or 1 otherwise \\

    \hline
    \multicolumn{4}{l}{~~~~\lst{with} $(Byte, v \in \Denot{Byte})$} \\
    \hline
    $~~~~~~v$  & \lst{Byte} & 1 &  in a single byte \\

    \hline
    \multicolumn{4}{l}{~~~~\lst{with} $(N, v \in \Denot{N}), N \in {Short, Int, Long}$} \\
    \hline
    $~~~~~~v$  & \lst{VLQ(ZigZag($$N$$))} & [1..3] & 
      16,32,64-bit signed integer encoded using \hyperref[sec:zigzag-encoding]{ZigZag} 
      and then \hyperref[sec:vlq-encoding]{VLQ} \\

    \hline
    \multicolumn{4}{l}{~~~~\lst{with} $(BigInt, v \in \Denot{BigInt})$} \\
    \multicolumn{4}{l}{~~~~~~$bytes = v$\lst{.toByteArray} } \\
    \hline
    $~~~~~~numBytes$  & \lst{VLQ(UInt)} &  & number of bytes in $bytes$ array \\
    \hline
    $~~~~~~bytes$  & \lst{Bytes} &  & serialized $bytes$ array \\

    \hline
    \multicolumn{4}{l}{~~~~\lst{with} $(GroupElement, v \in \Denot{GroupElement})$} \\
    \hline
    ~~~~~~$v$  & \lst{GroupElement} &  & serialization of GroupElement data. See~\ref{sec:ser:data:groupelement} \\

    \hline
    \multicolumn{4}{l}{~~~~\lst{with} $(SigmaProp, v \in \Denot{SigmaProp})$} \\
    \hline
    ~~~~~~$v$  & \lst{SigmaProp} &  & serialization of SigmaProp data. See~\ref{sec:ser:data:sigmaprop} \\

    \hline
    \multicolumn{4}{l}{~~~~\lst{with} $(Box, v \in \Denot{Box})$} \\
    \hline
    ~~~~~~$v$  & \lst{Box} &  & serialization of Box data. See~\ref{sec:ser:data:box} \\

    \hline
    \multicolumn{4}{l}{~~~~\lst{with} $(AvlTree, v \in \Denot{AvlTree})$} \\
    \hline
    ~~~~~~$v$  & \lst{AvlTree} &  & serialization of AvlTree data. See~\ref{sec:ser:data:avltree} \\

    \hline
    \multicolumn{4}{l}{~~~~\lst{with} $(Coll[T], v \in \Denot{Coll[T]})$} \\
    \hline
    $~~~~~~len$  & \lst{VLQ(UShort)} & [1..3] & length of the collection \\
    \hline
    \multicolumn{4}{l}{~~~~~~\lst{match} $(T, v)$ } \\

    \multicolumn{4}{l}{~~~~~~~~\lst{with} $(Boolean, v \in \Denot{Coll[Boolean]})$} \\
    \hline
    $~~~~~~~~~~v$  & \lst{Bits} & [1..1024] & boolean values packed in bits \\
    \hline

    \multicolumn{4}{l}{~~~~~~~~\lst{with} $(Byte, v \in \Denot{Coll[Byte]})$} \\
    \hline
    $~~~~~~~~~~v$  & \lst{Bytes} & $[1..len]$ & items of the collection  \\
    \hline
    \multicolumn{4}{l}{~~~~~~~~\lst{otherwise} } \\
    \multicolumn{4}{l}{~~~~~~~~~~\lst{for}~$i=1$~\lst{to}~$len$~\lst{do}~\lst{serializeData(}$T, v_i$\lst{) end for}} \\
    \multicolumn{4}{l}{~~~~~~\lst{end match}} \\
    \multicolumn{4}{l}{~~\lst{end match}} \\
    \multicolumn{4}{l}{\lst{end serializeData}} \\
    \hline
    \hline
\end{tabularx}\)
\end{figure}

\subsubsection{GroupElement serialization}
\label{sec:ser:data:groupelement}

A value of the \lst{GroupElement} type is represented in reference implementation using
\lst{SecP256K1Point} class of the \lst{org.bouncycastle.math.ec.custom.sec} package and
serialized into ASN.1 encoding. During deserialization the different encodings are
taken into account including point compression for $F_p$ (see X9.62 sec. 4.2.1 pg. 17).

\begin{figure}[H] \footnotesize
\caption{GroupElement serialization format}\vspace{-7pt}
\label{fig:ser:data:groupelement}
\(\begin{tabularx}{\textwidth}{| l | l | l | X |}
    \hline
    \bf{Slot} & \bf{Format} & \bf{\#bytes} & \bf{Description} \\
    \hline
    \multicolumn{4}{l}{\lst{def serialize(}$ge$\lst{)}} \\
    \multicolumn{4}{l}{~~\lst{if} $ge.isInfinity$ \lst{then}} \\
    \hline
    ~~~~$bytes$  & \lst{rep(}$0, 33$\lst{)} & $ 33 $ & all bytes = 0 \\ 
    \hline
    \multicolumn{4}{l}{~~\lst{else}} \\
    \hline
    ~~~~$bytes$  & Bytes & $33$ & where $bytes(0) \neq 0$, see \lst{sigmastate.serialization.GroupElementSerializer} \\ 
    \hline
    \multicolumn{4}{l}{~~\lst{end if}} \\
    \multicolumn{4}{l}{\lst{end def}} \\
    \hline
    \hline
\end{tabularx}\)
\end{figure}

\subsubsection{SigmaProp serialization}
\label{sec:ser:data:sigmaprop}

In reference implementation values of \lst{SigmaProp} type are serialized using 
\lst{SigmaBoolean.serializer}

\begin{figure}[H] \footnotesize
\caption{SigmaProp serialization format}\vspace{-7pt}
\label{fig:ser:data:sigmaprop}
\(\begin{tabularx}{\textwidth}{| l | l | l | X |}
    \hline
    \bf{Slot} & \bf{Format} & \bf{\#bytes} & \bf{Description} \\
    \hline
    \multicolumn{4}{l}{\lst{def serializeSigma(}$sp: SigmaTree$\lst{)}} \\
    \hline
    ~~$sp.opCode$  & \lst{Byte} & $1$ & opcode of SigmaTree node\\
    \hline
    \multicolumn{4}{l}{~~\lst{match} $sp$} \\
    \multicolumn{4}{l}{~~~~\lst{with} $dl: ProveDlog$} \\
        \hline
        ~~~~~~$dl.value$  & \lst{GroupElement} & $33$ & see \ref{sec:ser:data:groupelement} \\
        \hline
    \multicolumn{4}{l}{~~~~\lst{with} $dht: ProveDHTuple$} \\
        \hline
        ~~~~~~$dht.gv$  & \lst{GroupElement} & $33$ & see \ref{sec:ser:data:groupelement} \\
        \hline
        ~~~~~~$dht.hv$  & \lst{GroupElement} & $33$ &  \\
        \hline
        ~~~~~~$dht.uv$  & \lst{GroupElement} & $33$ &  \\    
        \hline
        ~~~~~~$dht.vv$  & \lst{GroupElement} & $33$ &  \\
        \hline
    \multicolumn{4}{l}{~~~~\lst{with} $and: CAND$} \\
        \hline
        ~~~~~~$nChildren$  & \lst{VLQ(UShort)} & $1..3$ & number of children \\
        \hline
        \multicolumn{4}{l}{~~~~~~\lst{for}~$i=1$~\lst{to}~$nChildren$~\lst{do}~\lst{serializeSigma(}$and.$\lst{children(}$i$\lst{)) end for}} \\
    \multicolumn{4}{l}{~~~~\lst{with} $or: COR$} \\
        \hline
        ~~~~~~$nChildren$  & \lst{VLQ(UShort)} & $1..3$ & number of children \\
        \hline
        \multicolumn{4}{l}{~~~~~~\lst{for}~$i=1$~\lst{to}~$nChildren$~\lst{do}~\lst{serializeSigma(}$or.$\lst{children(}$i$\lst{)) end for}} \\
    \multicolumn{4}{l}{~~~~\lst{with} $th: CTHRESHOLD$} \\
        \hline
        ~~~~~~$th.k$       & \lst{VLQ(UShort)} & $1..3$ & $k$ out of $n$ \\
        \hline
        ~~~~~~$nChildren$  & \lst{VLQ(UShort)} & $1..3$ & number of children \\
        \hline
        \multicolumn{4}{l}{~~~~~~\lst{for}~$i=1$~\lst{to}~$nChildren$~\lst{do}~\lst{serializeSigma(}$th.$\lst{children(}$i$\lst{)) end for}} \\
    \multicolumn{4}{l}{~~~~\lst{with} $\_: TrivialProp$ \lst{// besides opCode no additional bytes}} \\
    \multicolumn{4}{l}{~~\lst{end match}} \\
    \multicolumn{4}{l}{\lst{end def}} \\
    \hline
    \hline
\end{tabularx}\)
\end{figure}

\subsubsection{AvlTree serialization}
\label{sec:ser:data:avltree}

In reference implementation values of \lst{AvlTree} type are serialized using 
\lst{AvlTreeData.serializer}.

\begin{figure}[H] \footnotesize
\caption{AvlTree serialization format}\vspace{-7pt}
\label{fig:ser:data:avltree}
\(\begin{tabularx}{\textwidth}{| l | l | l | X |}
    \hline
    \bf{Slot} & \bf{Format} & \bf{\#bytes} & \bf{Description} \\
    \hline
    $digest$  & \lst{Bytes} & $DigestSize$ & authenticated tree digest: root hash along with tree height \\    
    \hline
    $treeFlags$  & \lst{UByte} & $1$ & 
        allowed modifications of the tree. 
        The operation is allowed when bit is set to 1. 
        bit0 - insert, bit1 - update, bit2 - remove \\
    \hline
    $keyLength$  & \lst{VLQ(UInt)} & $[1..5]$ & the length of each key in the tree \\    
    \hline
    \multicolumn{4}{l}{\lst{optional}~$valueLength$} \\
    \hline
    ~~$tag$ & \lst{Byte} & 1 & 0 - no value; 1 - has value \\
    \hline
    \multicolumn{4}{l}{~~\lst{when}~$tag == 1$} \\
    \hline
    ~~~~ $valueLength$ & \lst{VLQ(UInt)} & [1, *] & the length of all the values in the tree \\
    \hline
    \multicolumn{4}{l}{\lst{end optional}} \\
    \hline
    \hline
\end{tabularx}\)
\end{figure}




\subsection{Constant Serialization}
\label{sec:ser:const}

\lst{Constant} format is simple and self sufficient to represent any data value.
Serialized bytes of the \lst{Constant} format contain both the type bytes and the data
bytes, thus it can be stored or wire transfered and then later unambiguously
interpreted. The format is shown in Figure~\ref{fig:ser:const}

\begin{figure}[h] \footnotesize
\caption{Constant serialization format}\vspace{-7pt}
\label{fig:ser:const}
\(\begin{tabularx}{\textwidth}{| l | l | l | X |}
    \hline
    \bf{Slot} & \bf{Format} & \bf{\#bytes} & \bf{Description} \\
    \hline
    $type$  & \lst{Type} & $[1..\MaxTypeSize]$ & type of the data instance (see~\ref{sec:ser:type}) \\
    \hline
    $value$  & \lst{Data} & $[1..\MaxDataSize]$ & serialized data instance (see~\ref{sec:ser:data}) \\
    \hline
\end{tabularx}\)
\end{figure}

In order to parse the \lst{Constant} format first use type serializer form
section~\ref{sec:ser:type} and read the type. Then use the parsed type as an argument
of data serializer given in section~\ref{sec:ser:data}.

\subsection{Expression Serialization}
\label{sec:ser:expr}

Expressions of \langname are serialized as tree data structure using recursive
procedure described in Figure~\ref{fig:ser:expr}. Expression nodes are represented in
the reference implementation using \lst{Value} class.

\begin{figure}[h] \footnotesize
\caption{\lst{Expr} serialization format}\vspace{-7pt}
\label{fig:ser:expr}
\(\begin{tabularx}{\textwidth}{| l | l | l | X |}
    \hline
    \bf{Slot} & \bf{Format} & \bf{\#bytes} & \bf{Description} \\
    \hline
    \multicolumn{4}{l}{\lst{def serializeExpr(}$e$\lst{)}} \\
    \hline
    ~~$e.opCode$  & \lst{Byte} & $1$ & opcode of ErgoTree node, 
    used for selection of an appropriate node serializer from Appendix~\ref{sec:appendix:ergotree_serialization} \\
    \hline
    \multicolumn{4}{l}{~~\lst{if} $opCode <= $ \lst{LastConstantCode then}} \\
    \hline
    ~~~~$c$  & \lst{Constant} & $[1..\MaxConstSize]$ & Constant serializaton according to~\ref{sec:ser:const} \\ 
    \hline
    \multicolumn{4}{l}{~~\lst{else}} \\
    \hline
    ~~~~$body$  & \lst{Op} & $[1..\MaxExprSize]$ & serialization of the operation 
    depending on $e.opCode$ as defined in Appendix~\ref{sec:appendix:ergotree_serialization} \\ 
    \hline
    \multicolumn{4}{l}{~~\lst{end if}} \\
    \multicolumn{4}{l}{\lst{end serializeExpr}} \\
    \hline
    \hline
\end{tabularx}\)
\end{figure}


\subsection{\ASDag~serialization}
\label{sec:ser:ergotree}

The \langname propostions stored in UTXO boxes are represented in the reference
implementation using \lst{ErgoTree} class. The class is serialized using the \lst{ErgoTree}
serialization format  shown in Figure~\ref{fig:ser:ergotree}. 

\begin{figure}[h] \footnotesize
\caption{\ASDag serialization format}\vspace{-7pt}
\label{fig:ser:ergotree}
\(\begin{tabularx}{\textwidth}{| l | l | l | X |}
  \hline
  \bf{Slot} & \bf{Format} & \bf{\#bytes} & \bf{Description} \\
  \hline
  $ header $ & \lst{VLQ(UInt)} & [1, *] & the first bytes of serialized byte array which
  determines interpretation of the rest of the array \\
  \hline
  \multicolumn{4}{l}{\lst{if} $header[3] = 1$ \lst{then}} \\
  \hline
  ~~$size$ & \lst{VLQ(UInt)} & [1, *] & size of the constants and root, i.e. the number of bytes after $header$ and $size$ \\
  \hline
  \multicolumn{4}{l}{\lst{end for}} \\
  \hline
  $numConstants$ & \lst{VLQ(UInt)} & [1, *] & size of $constants$ array \\
  \hline
  \multicolumn{4}{l}{\lst{for}~$i=0$~\lst{to}~$numConstants-1$} \\
  \hline
      ~~ $ const_i $ & \lst{Const} & [1, *] & constant in i-th position \\
  \hline
  \multicolumn{4}{l}{\lst{end for}} \\
  \hline
  $ root $ & \lst{Expr} & [1, *] & 
    If $header[4] = 1$, the $root$ tree may contain ConstantPlaceholder
    nodes instead of Constant nodes (and may by only some of them, not all).
    Otherwise (i.e. if $header[4] = 0$) the root cannot contain placeholders (exception
    should be thrown). It is possible to have both constants and placeholders in the
    tree, but for every placeholder there should be a constant in $constants$ array of
    ErgoTree instance. \\
  \hline
\end{tabularx}\)
\end{figure}

Serialized instances of \lst{ErgoTree} class are self sufficient and can be stored and
passed around. \lst{ErgoTree} format defines top-level serialization format of
\langname scripts. The interpretation of the byte array depend on the first $header$
bytes, which uses VLQ encoding up to 30 bits. Currently we define meaning for only
first byte, which may be extended in future versions. The meaning of the bits is shown
in Figure~\ref{fig:ergotree:header}.

\begin{figure}[h] \footnotesize
\caption{\ASDag $header$ bits}\vspace{-7pt}
\label{fig:ergotree:header}
\(\begin{tabularx}{\textwidth}{| l | l | X |}
    \hline
    \bf{Bits} & \bf{Default} & \bf{Description} \\
    \hline
    Bits 0-2 & 0 & language version (current version == 0) \\
    \hline
    Bit 3 & 0 & $= 1$ if size of the whole tree is serialized after the header byte \\
    \hline
    Bit 4 & 0 & $= 1$ if constant segregation is used for this ErgoTree \\
    \hline
    Bit 5 & 0 & $= 1$ - reserved for context dependent costing (should be = 0) \\
    \hline
    Bit 6 & 0 & reserved for GZIP compression (should be 0) \\
    \hline
    Bit 7 & 0 & $= 1$ if the header contains more than 1 byte (should be 0) \\
    \hline
\end{tabularx}\)
\end{figure}

Currently we don't specify interpretation for the second and other bytes of
the header. We reserve the possibility to extend header by using Bit 7 == 1
and chain additional bytes as in VLQ. Once the new bytes are required, a new
version of the language should be created and implemented via
soft-forkability. That new language will give an interpretation for the new
bytes.

The default behavior of ErgoTreeSerializer is to preserve original structure
of \ASDag and check consistency. In case of any inconsistency the
serializer throws exception.

If constant segregation Bit4 is set to 1 then $constants$ collection contains
the constants for which there may be \lst{ConstantPlaceholder} nodes in the
tree. Nowever, if the constant segregation bit is 0, then $constants$
collection should be empty and any placeholder in the tree will lead to
exception.


