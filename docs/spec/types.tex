\section{Typing}
\label{sec:typing}

\langname is a strictly typed language, in which every term should have a
type in order to be wellformed and evaluated. Typing judgement of the form
$\Der{\Gamma}{e : T}$ say that $e$ is a term of type $T$ in the typing
context $\Gamma$.

\begin{figure}[h]

\begin{center}
% const, var, tuple
\(\begin{array}{c c c}
	\frac{}{\Der{\Gamma}{C(\_, T)~:~T}}~\lst{(Const)}
	& 
	\frac{}{\Der{\Gamma,x~:~T}{x~:~T}}~\lst{(Var)}
	&
	\frac{
		\Ov{\DerEnv{e_i:~T_i}}~~
		ptype(\delta, \Ov{T_i}) :~(T_1,\dots,T_n) \to T
	}{
		\Apply{\delta}{\Ov{e_i}}:~T
	}~\lst{(Prim)} \\
	& & \\ % blank line
\end{array}\) 


% tuples
\(\begin{array}{c}
\frac{\DerEnv{e_1 :~T_1}~~\dots~~\DerEnv{e_n :~T_n}}
     {\DerEnv{(e_1,\dots,e_n)~:~(T_1,\dots,T_n)}}~\lst{(Tuple)} \\
\\ % blank line
\end{array}\) 

% MethodCall
\(\begin{array}{c}
\frac{
		\DerEnv{e~:~I,~e_i:~T_i}~~
		mtype(m, I, \Ov{T_i})~:~(I, T_1,\dots,T_n) \to T
	}
	{ \Apply{e.m}{\Ov{e_i}}:~T }~\lst{(MethodCall)} \\
\\ % blank line
\end{array}\) 

% functions
\(\begin{array}{c c}
	\frac{\Der{\TEnv,\Ov{x_i:~T_i}}{e~:~T}}
	     {\Der{\Gamma}{\TyLam{x_i}{T_i}{e}~:~(T_0,\dots,T_n) \to T}}~\lst{(FuncValue)}
		  & 
	\frac{ \Der{\TEnv}{e_f:~(T_1,\dots,T_n) \to T}~~~\Ov{\Der{\TEnv}{e_i:~T_i}} }
		 { \Der{\Gamma}{\Apply{e_f}{\Ov{e_i}}~:~T} }~\lst{(Apply)} \\
& \\ % blank line
\end{array}\) 

% if
\(\begin{array}{c c}
	\frac{ \DerEnv{e_{cond} :~\lst{Boolean}}~~\DerEnv{e_1 :~T}~~\DerEnv{e_2 :~T} }
		{ \DerEnv{\IfThenElse{e_{cond}}{e_1}{e_2}~:~T} }~\lst{(If)}
		 & 
		 \\
		 & \\ % blank line
\end{array}\) 
\(
	\frac{ 
		\DerEnv{e_1 :~T_1}~\wedge~
		\forall k\in\{2,\dots,n\}~\Der{\Gamma,x_1:~T_1,\dots,x_{k-1}:~T_{k-1}}{e_k:~T_k}~\wedge~
        \Der{\Gamma,x_1:~T_1,\dots,x_n:~T_n}{e:~T}
		}
		{ \DerEnv{\{ \Ov{\text{\lst{val}}~x_i = e_i;}~e\}~:~T} }~\lst{(BlockValue)}
\)
% let
% \(
% \frac{\Der{\TEnv,x : T_1}{e_2 : T_2}}{\Der{\Gamma}{\Let{x}{e_1}{e_2} : T_2}}
% \)


\end{center}



\caption{Typing rules of \langname}
\label{fig:typing}
\end{figure}

Note that each well-typed term has exactly one type hence we assume there
exists a funcion $termType: Term \to \mathcal{T}$ which relates each well-typed
term with the corresponding type.

Primitive operations can be parameterized with type variables, for example
addition (\ref{sec:appendix:primops:Plus}) has the signature \lst{def +}$[$\lst{T}$]$(\lst{left}$:$~\lst{T}, \lst{right}$:$~\lst{T}): \lst{T}
where \lst{T} is one of the numeric types (Table~\ref{table:predeftypes}). 
Function $ptype$ returns the type of a primitive operation specialized for the concrete
types of its arguments, for example
$ptype(+,\lst{Int}, \lst{Int}) = (\lst{Int}, \lst{Int}) \to \lst{Int}$.

Similarily, the function $mtype$ returns a type of method specialized for concrete types of the arguments of the \lst{MethodCall} term.

\lst{BlockValue} rule defines a type of well-formed block expression. It
assumes a total ordering on \lst{val} definitions. If a block expression is
not well-formed than it cannot be typed and evaluated.

The rest of the rules are standard for typed lambda calculus.