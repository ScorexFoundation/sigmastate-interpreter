\subsection{\ASDag~serialization}
\label{sec:ser:ergotree}

The root of a serializable \langname term is a data structure called \ASDag
which serialization format shown in Figure~\ref{fig:ergotree}

\begin{figure}[h]
\footnotesize
\(\begin{tabularx}{\textwidth}{| l | l | l | X |}
  \hline
  \bf{Slot} & \bf{Format} & \bf{\#bytes} & \bf{Description} \\
  \hline
  $ header $ & \lst{VLQ(UInt)} & [1, *] & the first bytes of serialized byte array which
  determines interpretation of the rest of the array \\
  \hline
  $numConstants$ & \lst{VLQ(UInt)} & [1, *] & size of $constants$ array \\
  \hline
  \multicolumn{4}{l}{\lst{for}~$i=1$~\lst{to}~$numConstants$} \\
  \hline
      ~~ $ const_i $ & \lst{Const} & [1, *] & constant in i-th position \\
  \hline
  \multicolumn{4}{l}{\lst{end for}} \\
  \hline
  $ root $ & \lst{Expr} & [1, *] & If constantSegregationFlag is true, the  contains ConstantPlaceholder instead of some Constant nodes.
                       Otherwise may not contain placeholders.
                       It is possible to have both constants and placeholders in the tree, but for every placeholder
                       there should be a constant in $constants$ array. \\
  \hline
\end{tabularx}\)
\caption{\ASDag serialization format}
\label{fig:ser:ergotree}
\end{figure}


Serialized instances of \ASDag are self sufficient and can be stored and passed around.
\ASDag format defines top-level serialization format of \langname scripts.
The interpretation of the byte array depend on the first $header$ bytes, which uses VLQ encoding up to 30 bits.
Currently we define meaning for only first byte, which may be extended in future versions.

\begin{figure}[h]
    \footnotesize
\(\begin{tabularx}{\textwidth}{| l | l | X |}
    \hline
    \bf{Bits} & \bf{Default Value} & \bf{Description} \\
    \hline
    Bits 0-2 & 0 & language version (current version == 0) \\
    \hline
    Bit 3 & 0 & reserved (should be 0) \\
    \hline
    Bit 4 & 0 & == 1 if constant segregation is used for this ErgoTree (see~ Section~\ref{sec:ser:constant_segregation}\\
    \hline
    Bit 5 & 0 & == 1 - reserved for context dependent costing (should be = 0) \\
    \hline
    Bit 6 & 0 & reserved for GZIP compression (should be 0) \\
    \hline
    Bit 7 & 0 & == 1 if the header contains more than 1 byte (should be 0) \\
    \hline
\end{tabularx}\)
\caption{\ASDag $header$ bits}
\label{fig:ergotree:header}
\end{figure}

Currently we don't specify interpretation for the second and other bytes of
the header. We reserve the possibility to extend header by using Bit 7 == 1
and chain additional bytes as in VLQ. Once the new bytes are required, a new
version of the language should be created and implemented via
soft-forkability. That new language will give an interpretation for the new
bytes.

The default behavior of ErgoTreeSerializer is to preserve original structure
of \ASDag and check consistency. In case of any inconsistency the
serializer throws exception.

If constant segregation bit is set to 1 then $constants$ collection contains
the constants for which there may be \lst{ConstantPlaceholder} nodes in the
tree. If is however constant segregation bit is 0, then \lst{constants}
collection should be empty and any placeholder in the tree will lead to
exception.
