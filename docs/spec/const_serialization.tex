\subsection{Constant Serialization}
\label{sec:ser:const}

\lst{Constant} format is simple and self sufficient to represent any data value.
Serialized bytes of the \lst{Constant} format contain both the type bytes and the data
bytes, thus it can be stored or wire transfered and then later unambiguously
interpreted. The format is shown in Figure~\ref{fig:ser:const}

\begin{figure}[h] \footnotesize
\caption{Constant serialization format}\vspace{-7pt}
\label{fig:ser:const}
\(\begin{tabularx}{\textwidth}{| l | l | l | X |}
    \hline
    \bf{Slot} & \bf{Format} & \bf{\#bytes} & \bf{Description} \\
    \hline
    $type$  & \lst{Type} & $[1..\MaxTypeSize]$ & type of the data instance (see~\ref{sec:ser:type}) \\
    \hline
    $value$  & \lst{Data} & $[1..\MaxDataSize]$ & serialized data instance (see~\ref{sec:ser:data}) \\
    \hline
\end{tabularx}\)
\end{figure}

In order to parse the \lst{Constant} format first use type serializer form
section~\ref{sec:ser:type} and read the type. Then use the parsed type as an argument
of data serializer given in section~\ref{sec:ser:data}.