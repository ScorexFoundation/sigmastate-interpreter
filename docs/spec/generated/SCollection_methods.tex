
\subsubsection{\lst{SCollection.size} method (Code 12.1)}
\label{sec:type:SCollection:size}
\noindent
\begin{tabularx}{\textwidth}{| l | X |}
   \hline
   \bf{Description} & The size of the collection in elements. \\
  
  \hline
  \bf{Parameters} &
      \(\begin{array}{l l l}
         
      \end{array}\) \\
       
  \hline
  \bf{Result} & \lst{Int} \\
  \hline
  
  \bf{Serialized as} & \hyperref[sec:serialization:operation:SizeOf]{\lst{SizeOf}} \\
  \hline
       
\end{tabularx}



\subsubsection{\lst{SCollection.getOrElse} method (Code 12.2)}
\label{sec:type:SCollection:getOrElse}
\noindent
\begin{tabularx}{\textwidth}{| l | X |}
   \hline
   \bf{Description} & Return the element of collection if \lst{index} is in range \lst{0 .. size-1} \\
  
  \hline
  \bf{Parameters} &
      \(\begin{array}{l l l}
         \lst{index} & \lst{: Int} & \text{// index of the element of this collection} \\
\lst{default} & \lst{: IV} & \text{// value to return when \lst{index} is out of range} \\
      \end{array}\) \\
       
  \hline
  \bf{Result} & \lst{IV} \\
  \hline
  
  \bf{Serialized as} & \hyperref[sec:serialization:operation:ByIndex]{\lst{ByIndex}} \\
  \hline
       
\end{tabularx}



\subsubsection{\lst{SCollection.map} method (Code 12.3)}
\label{sec:type:SCollection:map}
\noindent
\begin{tabularx}{\textwidth}{| l | X |}
   \hline
   \bf{Description} &  Builds a new collection by applying a function to all elements of this collection.
 Returns a new collection of type \lst{Coll[B]} resulting from applying the given function
 \lst{f} to each element of this collection and collecting the results.
         \\
  
  \hline
  \bf{Parameters} &
      \(\begin{array}{l l l}
         \lst{f} & \lst{: (IV) => OV} & \text{// the function to apply to each element} \\
      \end{array}\) \\
       
  \hline
  \bf{Result} & \lst{Coll[OV]} \\
  \hline
  
  \bf{Serialized as} & \hyperref[sec:serialization:operation:MapCollection]{\lst{MapCollection}} \\
  \hline
       
\end{tabularx}



\subsubsection{\lst{SCollection.exists} method (Code 12.4)}
\label{sec:type:SCollection:exists}
\noindent
\begin{tabularx}{\textwidth}{| l | X |}
   \hline
   \bf{Description} & Tests whether a predicate holds for at least one element of this collection.
Returns \lst{true} if the given predicate \lst{p} is satisfied by at least one element of this collection, otherwise \lst{false}
         \\
  
  \hline
  \bf{Parameters} &
      \(\begin{array}{l l l}
         \lst{p} & \lst{: (IV) => Boolean} & \text{// the predicate used to test elements} \\
      \end{array}\) \\
       
  \hline
  \bf{Result} & \lst{Boolean} \\
  \hline
  
  \bf{Serialized as} & \hyperref[sec:serialization:operation:Exists]{\lst{Exists}} \\
  \hline
       
\end{tabularx}



\subsubsection{\lst{SCollection.fold} method (Code 12.5)}
\label{sec:type:SCollection:fold}
\noindent
\begin{tabularx}{\textwidth}{| l | X |}
   \hline
   \bf{Description} & Applies a binary operator to a start value and all elements of this collection, going left to right. \\
  
  \hline
  \bf{Parameters} &
      \(\begin{array}{l l l}
         \lst{zero} & \lst{: OV} & \text{// a starting value} \\
\lst{op} & \lst{: (OV,IV) => OV} & \text{// the binary operator} \\
      \end{array}\) \\
       
  \hline
  \bf{Result} & \lst{OV} \\
  \hline
  
  \bf{Serialized as} & \hyperref[sec:serialization:operation:Fold]{\lst{Fold}} \\
  \hline
       
\end{tabularx}



\subsubsection{\lst{SCollection.forall} method (Code 12.6)}
\label{sec:type:SCollection:forall}
\noindent
\begin{tabularx}{\textwidth}{| l | X |}
   \hline
   \bf{Description} & Tests whether a predicate holds for all elements of this collection.
Returns \lst{true} if this collection is empty or the given predicate \lst{p}
holds for all elements of this collection, otherwise \lst{false}.
         \\
  
  \hline
  \bf{Parameters} &
      \(\begin{array}{l l l}
         \lst{p} & \lst{: (IV) => Boolean} & \text{// the predicate used to test elements} \\
      \end{array}\) \\
       
  \hline
  \bf{Result} & \lst{Boolean} \\
  \hline
  
  \bf{Serialized as} & \hyperref[sec:serialization:operation:ForAll]{\lst{ForAll}} \\
  \hline
       
\end{tabularx}



\subsubsection{\lst{SCollection.slice} method (Code 12.7)}
\label{sec:type:SCollection:slice}
\noindent
\begin{tabularx}{\textwidth}{| l | X |}
   \hline
   \bf{Description} & Selects an interval of elements.  The returned collection is made up
  of all elements \lst{x} which satisfy the invariant:
  \lst{
     from <= indexOf(x) < until
  }
         \\
  
  \hline
  \bf{Parameters} &
      \(\begin{array}{l l l}
         \lst{from} & \lst{: Int} & \text{// the lowest index to include from this collection} \\
\lst{until} & \lst{: Int} & \text{// the lowest index to EXCLUDE from this collection} \\
      \end{array}\) \\
       
  \hline
  \bf{Result} & \lst{Coll[IV]} \\
  \hline
  
  \bf{Serialized as} & \hyperref[sec:serialization:operation:Slice]{\lst{Slice}} \\
  \hline
       
\end{tabularx}



\subsubsection{\lst{SCollection.filter} method (Code 12.8)}
\label{sec:type:SCollection:filter}
\noindent
\begin{tabularx}{\textwidth}{| l | X |}
   \hline
   \bf{Description} & Selects all elements of this collection which satisfy a predicate.
 Returns  a new collection consisting of all elements of this collection that satisfy the given
 predicate \lst{p}. The order of the elements is preserved.
         \\
  
  \hline
  \bf{Parameters} &
      \(\begin{array}{l l l}
         \lst{p} & \lst{: (IV) => Boolean} & \text{// the predicate used to test elements.} \\
      \end{array}\) \\
       
  \hline
  \bf{Result} & \lst{Coll[IV]} \\
  \hline
  
  \bf{Serialized as} & \hyperref[sec:serialization:operation:Filter]{\lst{Filter}} \\
  \hline
       
\end{tabularx}



\subsubsection{\lst{SCollection.append} method (Code 12.9)}
\label{sec:type:SCollection:append}
\noindent
\begin{tabularx}{\textwidth}{| l | X |}
   \hline
   \bf{Description} & Puts the elements of other collection after the elements of this collection (concatenation of 2 collections) \\
  
  \hline
  \bf{Parameters} &
      \(\begin{array}{l l l}
         \lst{other} & \lst{: Coll[IV]} & \text{// the collection to append at the end of this} \\
      \end{array}\) \\
       
  \hline
  \bf{Result} & \lst{Coll[IV]} \\
  \hline
  
  \bf{Serialized as} & \hyperref[sec:serialization:operation:Append]{\lst{Append}} \\
  \hline
       
\end{tabularx}



\subsubsection{\lst{SCollection.apply} method (Code 12.10)}
\label{sec:type:SCollection:apply}
\noindent
\begin{tabularx}{\textwidth}{| l | X |}
   \hline
   \bf{Description} & The element at given index.
 Indices start at \lst{0}; \lst{xs.apply(0)} is the first element of collection \lst{xs}.
 Note the indexing syntax \lst{xs(i)} is a shorthand for \lst{xs.apply(i)}.
 Returns the element at the given index.
 Throws an exception if \lst{i < 0} or \lst{length <= i}
         \\
  
  \hline
  \bf{Parameters} &
      \(\begin{array}{l l l}
         \lst{i} & \lst{: Int} & \text{// the index} \\
      \end{array}\) \\
       
  \hline
  \bf{Result} & \lst{IV} \\
  \hline
  
  \bf{Serialized as} & \hyperref[sec:serialization:operation:ByIndex]{\lst{ByIndex}} \\
  \hline
       
\end{tabularx}



\subsubsection{\lst{SCollection.indices} method (Code 12.14)}
\label{sec:type:SCollection:indices}
\noindent
\begin{tabularx}{\textwidth}{| l | X |}
   \hline
   \bf{Description} & Produces the range of all indices of this collection as a new collection
 containing [0 .. length-1] values.
         \\
  
  \hline
  \bf{Parameters} &
      \(\begin{array}{l l l}
         
      \end{array}\) \\
       
  \hline
  \bf{Result} & \lst{Coll[Int]} \\
  \hline
  
  \bf{Serialized as} & \hyperref[sec:serialization:operation:PropertyCall]{\lst{PropertyCall}} \\
  \hline
       
\end{tabularx}



\subsubsection{\lst{SCollection.flatMap} method (Code 12.15)}
\label{sec:type:SCollection:flatMap}
\noindent
\begin{tabularx}{\textwidth}{| l | X |}
   \hline
   \bf{Description} &  Builds a new collection by applying a function to all elements of this collection
 and using the elements of the resulting collections.
 Returns a new collection of type \lst{Coll[B]} resulting from applying the given collection-valued function
 \lst{f} to each element of this collection and concatenating the results.
         \\
  
  \hline
  \bf{Parameters} &
      \(\begin{array}{l l l}
         \lst{f} & \lst{: (IV) => Coll[OV]} & \text{// the function to apply to each element.} \\
      \end{array}\) \\
       
  \hline
  \bf{Result} & \lst{Coll[OV]} \\
  \hline
  
  \bf{Serialized as} & \hyperref[sec:serialization:operation:MethodCall]{\lst{MethodCall}} \\
  \hline
       
\end{tabularx}



\subsubsection{\lst{SCollection.patch} method (Code 12.19)}
\label{sec:type:SCollection:patch}
\noindent
\begin{tabularx}{\textwidth}{| l | X |}
   \hline
   \bf{Description} & Produces a new Coll where a slice of elements in this Coll is replaced by another Coll. \\
  
  \hline
  \bf{Parameters} &
      \(\begin{array}{l l l}
         
      \end{array}\) \\
       
  \hline
  \bf{Result} & \lst{Coll[IV]} \\
  \hline
  
  \bf{Serialized as} & \hyperref[sec:serialization:operation:MethodCall]{\lst{MethodCall}} \\
  \hline
       
\end{tabularx}



\subsubsection{\lst{SCollection.updated} method (Code 12.20)}
\label{sec:type:SCollection:updated}
\noindent
\begin{tabularx}{\textwidth}{| l | X |}
   \hline
   \bf{Description} & A copy of this Coll with one single replaced element. \\
  
  \hline
  \bf{Parameters} &
      \(\begin{array}{l l l}
         
      \end{array}\) \\
       
  \hline
  \bf{Result} & \lst{Coll[IV]} \\
  \hline
  
  \bf{Serialized as} & \hyperref[sec:serialization:operation:MethodCall]{\lst{MethodCall}} \\
  \hline
       
\end{tabularx}



\subsubsection{\lst{SCollection.updateMany} method (Code 12.21)}
\label{sec:type:SCollection:updateMany}
\noindent
\begin{tabularx}{\textwidth}{| l | X |}
   \hline
   \bf{Description} &  \\
  
  \hline
  \bf{Parameters} &
      \(\begin{array}{l l l}
         
      \end{array}\) \\
       
  \hline
  \bf{Result} & \lst{Coll[IV]} \\
  \hline
  
  \bf{Serialized as} & \hyperref[sec:serialization:operation:MethodCall]{\lst{MethodCall}} \\
  \hline
       
\end{tabularx}



\subsubsection{\lst{SCollection.indexOf} method (Code 12.26)}
\label{sec:type:SCollection:indexOf}
\noindent
\begin{tabularx}{\textwidth}{| l | X |}
   \hline
   \bf{Description} &  \\
  
  \hline
  \bf{Parameters} &
      \(\begin{array}{l l l}
         
      \end{array}\) \\
       
  \hline
  \bf{Result} & \lst{Int} \\
  \hline
  
  \bf{Serialized as} & \hyperref[sec:serialization:operation:MethodCall]{\lst{MethodCall}} \\
  \hline
       
\end{tabularx}



\subsubsection{\lst{SCollection.zip} method (Code 12.29)}
\label{sec:type:SCollection:zip}
\noindent
\begin{tabularx}{\textwidth}{| l | X |}
   \hline
   \bf{Description} &  \\
  
  \hline
  \bf{Parameters} &
      \(\begin{array}{l l l}
         
      \end{array}\) \\
       
  \hline
  \bf{Result} & \lst{Coll[(IV,OV)]} \\
  \hline
  
  \bf{Serialized as} & \hyperref[sec:serialization:operation:MethodCall]{\lst{MethodCall}} \\
  \hline
       
\end{tabularx}
