
\subsubsection{\lst{SCollection.size} method (Code 12.1)}
\noindent
\begin{tabularx}{\textwidth}{| l | X |}
   \hline
   \bf{Description} & The size of the collection in elements. \\
  
  \hline
  \bf{Parameters} &
      \(\begin{array}{l l l}
         
      \end{array}\) \\
       
  \hline
  \bf{Result} & \lst{Int} \\
  \hline
  
  \bf{Serialized as} & \lst{SizeOf(opCode=177)} \\
  \hline
       
\end{tabularx}



\subsubsection{\lst{SCollection.getOrElse} method (Code 12.2)}
\noindent
\begin{tabularx}{\textwidth}{| l | X |}
   \hline
   \bf{Description} & Return the element of collection if \lst{index} is in range \lst{0 .. size-1} \\
  
  \hline
  \bf{Parameters} &
      \(\begin{array}{l l l}
         \lst{index} & \lst{: Int} & \text{// index of the element of this collection} \\
\lst{default} & \lst{: IV} & \text{// value to return when \lst{index} is out of range} \\
      \end{array}\) \\
       
  \hline
  \bf{Result} & \lst{IV} \\
  \hline
  
  \bf{Serialized as} & \lst{ByIndex(opCode=178)} \\
  \hline
       
\end{tabularx}



\subsubsection{\lst{SCollection.map} method (Code 12.3)}
\noindent
\begin{tabularx}{\textwidth}{| l | X |}
   \hline
   \bf{Description} &  Builds a new collection by applying a function to all elements of this collection.
 Returns a new collection of type \lst{Coll[B]} resulting from applying the given function
 \lst{f} to each element of this collection and collecting the results.
         \\
  
  \hline
  \bf{Parameters} &
      \(\begin{array}{l l l}
         \lst{f} & \lst{: (IV) => OV} & \text{// the function to apply to each element} \\
      \end{array}\) \\
       
  \hline
  \bf{Result} & \lst{Coll[OV]} \\
  \hline
  
  \bf{Serialized as} & \lst{MapCollection(opCode=173)} \\
  \hline
       
\end{tabularx}



\subsubsection{\lst{SCollection.exists} method (Code 12.4)}
\noindent
\begin{tabularx}{\textwidth}{| l | X |}
   \hline
   \bf{Description} & Tests whether a predicate holds for at least one element of this collection.
Returns \lst{true} if the given predicate \lst{p} is satisfied by at least one element of this collection, otherwise \lst{false}
         \\
  
  \hline
  \bf{Parameters} &
      \(\begin{array}{l l l}
         \lst{p} & \lst{: (IV) => Boolean} & \text{// the predicate used to test elements} \\
      \end{array}\) \\
       
  \hline
  \bf{Result} & \lst{Boolean} \\
  \hline
  
  \bf{Serialized as} & \lst{Exists(opCode=174)} \\
  \hline
       
\end{tabularx}



\subsubsection{\lst{SCollection.fold} method (Code 12.5)}
\noindent
\begin{tabularx}{\textwidth}{| l | X |}
   \hline
   \bf{Description} & Applies a binary operator to a start value and all elements of this collection, going left to right. \\
  
  \hline
  \bf{Parameters} &
      \(\begin{array}{l l l}
         \lst{zero} & \lst{: OV} & \text{// a starting value} \\
\lst{op} & \lst{: (OV,IV) => OV} & \text{// the binary operator} \\
      \end{array}\) \\
       
  \hline
  \bf{Result} & \lst{OV} \\
  \hline
  
  \bf{Serialized as} & \lst{Fold(opCode=176)} \\
  \hline
       
\end{tabularx}



\subsubsection{\lst{SCollection.forall} method (Code 12.6)}
\noindent
\begin{tabularx}{\textwidth}{| l | X |}
   \hline
   \bf{Description} & Tests whether a predicate holds for all elements of this collection.
Returns \lst{true} if this collection is empty or the given predicate \lst{p}
holds for all elements of this collection, otherwise \lst{false}.
         \\
  
  \hline
  \bf{Parameters} &
      \(\begin{array}{l l l}
         \lst{p} & \lst{: (IV) => Boolean} & \text{// the predicate used to test elements} \\
      \end{array}\) \\
       
  \hline
  \bf{Result} & \lst{Boolean} \\
  \hline
  
  \bf{Serialized as} & \lst{ForAll(opCode=175)} \\
  \hline
       
\end{tabularx}



\subsubsection{\lst{SCollection.slice} method (Code 12.7)}
\noindent
\begin{tabularx}{\textwidth}{| l | X |}
   \hline
   \bf{Description} & Selects an interval of elements.  The returned collection is made up
  of all elements \lst{x} which satisfy the invariant:
  \lst{
     from <= indexOf(x) < until
  }
         \\
  
  \hline
  \bf{Parameters} &
      \(\begin{array}{l l l}
         \lst{from} & \lst{: Int} & \text{// the lowest index to include from this collection} \\
\lst{until} & \lst{: Int} & \text{// the lowest index to EXCLUDE from this collection} \\
      \end{array}\) \\
       
  \hline
  \bf{Result} & \lst{Coll[IV]} \\
  \hline
  
  \bf{Serialized as} & \lst{Slice(opCode=180)} \\
  \hline
       
\end{tabularx}



\subsubsection{\lst{SCollection.filter} method (Code 12.8)}
\noindent
\begin{tabularx}{\textwidth}{| l | X |}
   \hline
   \bf{Description} & Selects all elements of this collection which satisfy a predicate.
 Returns  a new collection consisting of all elements of this collection that satisfy the given
 predicate \lst{p}. The order of the elements is preserved.
         \\
  
  \hline
  \bf{Parameters} &
      \(\begin{array}{l l l}
         \lst{p} & \lst{: (IV) => Boolean} & \text{// the predicate used to test elements.} \\
      \end{array}\) \\
       
  \hline
  \bf{Result} & \lst{Coll[IV]} \\
  \hline
  
  \bf{Serialized as} & \lst{Filter(opCode=181)} \\
  \hline
       
\end{tabularx}



\subsubsection{\lst{SCollection.append} method (Code 12.9)}
\noindent
\begin{tabularx}{\textwidth}{| l | X |}
   \hline
   \bf{Description} & Puts the elements of other collection after the elements of this collection (concatenation of 2 collections) \\
  
  \hline
  \bf{Parameters} &
      \(\begin{array}{l l l}
         \lst{other} & \lst{: Coll[IV]} & \text{// the collection to append at the end of this} \\
      \end{array}\) \\
       
  \hline
  \bf{Result} & \lst{Coll[IV]} \\
  \hline
  
  \bf{Serialized as} & \lst{Append(opCode=179)} \\
  \hline
       
\end{tabularx}



\subsubsection{\lst{SCollection.apply} method (Code 12.10)}
\noindent
\begin{tabularx}{\textwidth}{| l | X |}
   \hline
   \bf{Description} & The element at given index.
 Indices start at \lst{0}; \lst{xs.apply(0)} is the first element of collection \lst{xs}.
 Note the indexing syntax \lst{xs(i)} is a shorthand for \lst{xs.apply(i)}.
 Returns the element at the given index.
 Throws an exception if \lst{i < 0} or \lst{length <= i}
         \\
  
  \hline
  \bf{Parameters} &
      \(\begin{array}{l l l}
         \lst{i} & \lst{: Int} & \text{// the index} \\
      \end{array}\) \\
       
  \hline
  \bf{Result} & \lst{IV} \\
  \hline
  
  \bf{Serialized as} & \lst{ByIndex(opCode=178)} \\
  \hline
       
\end{tabularx}



\subsubsection{\lst{SCollection.<<} method (Code 12.11)}
\noindent
\begin{tabularx}{\textwidth}{| l | X |}
   \hline
   \bf{Description} &  \\
  
  \hline
  \bf{Parameters} &
      \(\begin{array}{l l l}
         \lst{arg0} & \lst{: Coll[IV]} & \text{// } \\
\lst{arg1} & \lst{: Int} & \text{// } \\
      \end{array}\) \\
       
  \hline
  \bf{Result} & \lst{Coll[IV]} \\
  \hline
  
\end{tabularx}



\subsubsection{\lst{SCollection.>>} method (Code 12.12)}
\noindent
\begin{tabularx}{\textwidth}{| l | X |}
   \hline
   \bf{Description} &  \\
  
  \hline
  \bf{Parameters} &
      \(\begin{array}{l l l}
         \lst{arg0} & \lst{: Coll[IV]} & \text{// } \\
\lst{arg1} & \lst{: Int} & \text{// } \\
      \end{array}\) \\
       
  \hline
  \bf{Result} & \lst{Coll[IV]} \\
  \hline
  
\end{tabularx}



\subsubsection{\lst{SCollection.>>>} method (Code 12.13)}
\noindent
\begin{tabularx}{\textwidth}{| l | X |}
   \hline
   \bf{Description} &  \\
  
  \hline
  \bf{Parameters} &
      \(\begin{array}{l l l}
         \lst{arg0} & \lst{: Coll[Boolean]} & \text{// } \\
\lst{arg1} & \lst{: Int} & \text{// } \\
      \end{array}\) \\
       
  \hline
  \bf{Result} & \lst{Coll[Boolean]} \\
  \hline
  
\end{tabularx}



\subsubsection{\lst{SCollection.indices} method (Code 12.14)}
\noindent
\begin{tabularx}{\textwidth}{| l | X |}
   \hline
   \bf{Description} & Produces the range of all indices of this collection as a new collection
 containing [0 .. length-1] values.
         \\
  
  \hline
  \bf{Parameters} &
      \(\begin{array}{l l l}
         
      \end{array}\) \\
       
  \hline
  \bf{Result} & \lst{Coll[Int]} \\
  \hline
  
  \bf{Serialized as} & \lst{PropertyCall(opCode=219)} \\
  \hline
       
\end{tabularx}



\subsubsection{\lst{SCollection.flatMap} method (Code 12.15)}
\noindent
\begin{tabularx}{\textwidth}{| l | X |}
   \hline
   \bf{Description} &  Builds a new collection by applying a function to all elements of this collection
 and using the elements of the resulting collections.
 Function \lst{f} is constrained to be of the form \lst{x => x.someProperty}, otherwise
 it is illegal.
 Returns a new collection of type \lst{Coll[B]} resulting from applying the given collection-valued function
 \lst{f} to each element of this collection and concatenating the results.
         \\
  
  \hline
  \bf{Parameters} &
      \(\begin{array}{l l l}
         \lst{f} & \lst{: (IV) => Coll[OV]} & \text{// the function to apply to each element.} \\
      \end{array}\) \\
       
  \hline
  \bf{Result} & \lst{Coll[OV]} \\
  \hline
  
  \bf{Serialized as} & \lst{MethodCall(opCode=220)} \\
  \hline
       
\end{tabularx}



\subsubsection{\lst{SCollection.segmentLength} method (Code 12.16)}
\noindent
\begin{tabularx}{\textwidth}{| l | X |}
   \hline
   \bf{Description} & Computes length of longest segment whose elements all satisfy some predicate.
 Returns the length of the longest segment of this collection starting from index \lst{from}
 such that every element of the segment satisfies the predicate \lst{p}.
         \\
  
  \hline
  \bf{Parameters} &
      \(\begin{array}{l l l}
         \lst{p} & \lst{: (IV) => Boolean} & \text{// the predicate used to test elements.} \\
\lst{from} & \lst{: Int} & \text{// the index where the search starts.} \\
      \end{array}\) \\
       
  \hline
  \bf{Result} & \lst{Int} \\
  \hline
  
  \bf{Serialized as} & \lst{MethodCall(opCode=220)} \\
  \hline
       
\end{tabularx}



\subsubsection{\lst{SCollection.indexWhere} method (Code 12.17)}
\noindent
\begin{tabularx}{\textwidth}{| l | X |}
   \hline
   \bf{Description} & Finds index of the first element satisfying some predicate after or at some start index.
 Returns the index \lst{>= from} of the first element of this collection that satisfies the predicate \lst{p},
 or \lst{-1}, if none exists.
         \\
  
  \hline
  \bf{Parameters} &
      \(\begin{array}{l l l}
         \lst{p} & \lst{: (IV) => Boolean} & \text{// the predicate used to test elements.} \\
\lst{from} & \lst{: Int} & \text{// the start index} \\
      \end{array}\) \\
       
  \hline
  \bf{Result} & \lst{Int} \\
  \hline
  
  \bf{Serialized as} & \lst{MethodCall(opCode=220)} \\
  \hline
       
\end{tabularx}



\subsubsection{\lst{SCollection.lastIndexWhere} method (Code 12.18)}
\noindent
\begin{tabularx}{\textwidth}{| l | X |}
   \hline
   \bf{Description} & Finds index of last element satisfying some predicate before or at given end index.
 Return the index \lst{<= end} of the last element of this collection that satisfies the predicate \lst{p},
 or \lst{-1}, if none exists.
         \\
  
  \hline
  \bf{Parameters} &
      \(\begin{array}{l l l}
         \lst{p} & \lst{: (IV) => Boolean} & \text{// the predicate used to test elements.} \\
      \end{array}\) \\
       
  \hline
  \bf{Result} & \lst{Int} \\
  \hline
  
  \bf{Serialized as} & \lst{MethodCall(opCode=220)} \\
  \hline
       
\end{tabularx}



\subsubsection{\lst{SCollection.patch} method (Code 12.19)}
\noindent
\begin{tabularx}{\textwidth}{| l | X |}
   \hline
   \bf{Description} &  \\
  
  \hline
  \bf{Parameters} &
      \(\begin{array}{l l l}
         
      \end{array}\) \\
       
  \hline
  \bf{Result} & \lst{Coll[IV]} \\
  \hline
  
  \bf{Serialized as} & \lst{MethodCall(opCode=220)} \\
  \hline
       
\end{tabularx}



\subsubsection{\lst{SCollection.updated} method (Code 12.20)}
\noindent
\begin{tabularx}{\textwidth}{| l | X |}
   \hline
   \bf{Description} &  \\
  
  \hline
  \bf{Parameters} &
      \(\begin{array}{l l l}
         
      \end{array}\) \\
       
  \hline
  \bf{Result} & \lst{Coll[IV]} \\
  \hline
  
  \bf{Serialized as} & \lst{MethodCall(opCode=220)} \\
  \hline
       
\end{tabularx}



\subsubsection{\lst{SCollection.updateMany} method (Code 12.21)}
\noindent
\begin{tabularx}{\textwidth}{| l | X |}
   \hline
   \bf{Description} &  \\
  
  \hline
  \bf{Parameters} &
      \(\begin{array}{l l l}
         
      \end{array}\) \\
       
  \hline
  \bf{Result} & \lst{Coll[IV]} \\
  \hline
  
  \bf{Serialized as} & \lst{MethodCall(opCode=220)} \\
  \hline
       
\end{tabularx}



\subsubsection{\lst{SCollection.unionSets} method (Code 12.22)}
\noindent
\begin{tabularx}{\textwidth}{| l | X |}
   \hline
   \bf{Description} &  \\
  
  \hline
  \bf{Parameters} &
      \(\begin{array}{l l l}
         
      \end{array}\) \\
       
  \hline
  \bf{Result} & \lst{Coll[IV]} \\
  \hline
  
  \bf{Serialized as} & \lst{MethodCall(opCode=220)} \\
  \hline
       
\end{tabularx}



\subsubsection{\lst{SCollection.diff} method (Code 12.23)}
\noindent
\begin{tabularx}{\textwidth}{| l | X |}
   \hline
   \bf{Description} &  \\
  
  \hline
  \bf{Parameters} &
      \(\begin{array}{l l l}
         
      \end{array}\) \\
       
  \hline
  \bf{Result} & \lst{Coll[IV]} \\
  \hline
  
  \bf{Serialized as} & \lst{MethodCall(opCode=220)} \\
  \hline
       
\end{tabularx}



\subsubsection{\lst{SCollection.intersect} method (Code 12.24)}
\noindent
\begin{tabularx}{\textwidth}{| l | X |}
   \hline
   \bf{Description} &  \\
  
  \hline
  \bf{Parameters} &
      \(\begin{array}{l l l}
         
      \end{array}\) \\
       
  \hline
  \bf{Result} & \lst{Coll[IV]} \\
  \hline
  
  \bf{Serialized as} & \lst{MethodCall(opCode=220)} \\
  \hline
       
\end{tabularx}



\subsubsection{\lst{SCollection.prefixLength} method (Code 12.25)}
\noindent
\begin{tabularx}{\textwidth}{| l | X |}
   \hline
   \bf{Description} &  \\
  
  \hline
  \bf{Parameters} &
      \(\begin{array}{l l l}
         
      \end{array}\) \\
       
  \hline
  \bf{Result} & \lst{Int} \\
  \hline
  
  \bf{Serialized as} & \lst{MethodCall(opCode=220)} \\
  \hline
       
\end{tabularx}



\subsubsection{\lst{SCollection.indexOf} method (Code 12.26)}
\noindent
\begin{tabularx}{\textwidth}{| l | X |}
   \hline
   \bf{Description} &  \\
  
  \hline
  \bf{Parameters} &
      \(\begin{array}{l l l}
         
      \end{array}\) \\
       
  \hline
  \bf{Result} & \lst{Int} \\
  \hline
  
  \bf{Serialized as} & \lst{MethodCall(opCode=220)} \\
  \hline
       
\end{tabularx}



\subsubsection{\lst{SCollection.lastIndexOf} method (Code 12.27)}
\noindent
\begin{tabularx}{\textwidth}{| l | X |}
   \hline
   \bf{Description} &  \\
  
  \hline
  \bf{Parameters} &
      \(\begin{array}{l l l}
         
      \end{array}\) \\
       
  \hline
  \bf{Result} & \lst{Int} \\
  \hline
  
  \bf{Serialized as} & \lst{MethodCall(opCode=220)} \\
  \hline
       
\end{tabularx}



\subsubsection{\lst{SCollection.find} method (Code 12.28)}
\noindent
\begin{tabularx}{\textwidth}{| l | X |}
   \hline
   \bf{Description} &  \\
  
  \hline
  \bf{Parameters} &
      \(\begin{array}{l l l}
         
      \end{array}\) \\
       
  \hline
  \bf{Result} & \lst{Option[IV]} \\
  \hline
  
  \bf{Serialized as} & \lst{MethodCall(opCode=220)} \\
  \hline
       
\end{tabularx}



\subsubsection{\lst{SCollection.zip} method (Code 12.29)}
\noindent
\begin{tabularx}{\textwidth}{| l | X |}
   \hline
   \bf{Description} &  \\
  
  \hline
  \bf{Parameters} &
      \(\begin{array}{l l l}
         
      \end{array}\) \\
       
  \hline
  \bf{Result} & \lst{Coll[(IV,OV)]} \\
  \hline
  
  \bf{Serialized as} & \lst{MethodCall(opCode=220)} \\
  \hline
       
\end{tabularx}



\subsubsection{\lst{SCollection.distinct} method (Code 12.30)}
\noindent
\begin{tabularx}{\textwidth}{| l | X |}
   \hline
   \bf{Description} &  \\
  
  \hline
  \bf{Parameters} &
      \(\begin{array}{l l l}
         
      \end{array}\) \\
       
  \hline
  \bf{Result} & \lst{Coll[IV]} \\
  \hline
  
  \bf{Serialized as} & \lst{PropertyCall(opCode=219)} \\
  \hline
       
\end{tabularx}



\subsubsection{\lst{SCollection.startsWith} method (Code 12.31)}
\noindent
\begin{tabularx}{\textwidth}{| l | X |}
   \hline
   \bf{Description} &  \\
  
  \hline
  \bf{Parameters} &
      \(\begin{array}{l l l}
         
      \end{array}\) \\
       
  \hline
  \bf{Result} & \lst{Boolean} \\
  \hline
  
  \bf{Serialized as} & \lst{MethodCall(opCode=220)} \\
  \hline
       
\end{tabularx}



\subsubsection{\lst{SCollection.endsWith} method (Code 12.32)}
\noindent
\begin{tabularx}{\textwidth}{| l | X |}
   \hline
   \bf{Description} &  \\
  
  \hline
  \bf{Parameters} &
      \(\begin{array}{l l l}
         
      \end{array}\) \\
       
  \hline
  \bf{Result} & \lst{Boolean} \\
  \hline
  
  \bf{Serialized as} & \lst{MethodCall(opCode=220)} \\
  \hline
       
\end{tabularx}



\subsubsection{\lst{SCollection.partition} method (Code 12.33)}
\noindent
\begin{tabularx}{\textwidth}{| l | X |}
   \hline
   \bf{Description} &  \\
  
  \hline
  \bf{Parameters} &
      \(\begin{array}{l l l}
         
      \end{array}\) \\
       
  \hline
  \bf{Result} & \lst{(Coll[IV],Coll[IV])} \\
  \hline
  
  \bf{Serialized as} & \lst{MethodCall(opCode=220)} \\
  \hline
       
\end{tabularx}



\subsubsection{\lst{SCollection.mapReduce} method (Code 12.34)}
\noindent
\begin{tabularx}{\textwidth}{| l | X |}
   \hline
   \bf{Description} &  \\
  
  \hline
  \bf{Parameters} &
      \(\begin{array}{l l l}
         
      \end{array}\) \\
       
  \hline
  \bf{Result} & \lst{Coll[(K,V)]} \\
  \hline
  
  \bf{Serialized as} & \lst{MethodCall(opCode=220)} \\
  \hline
       
\end{tabularx}
