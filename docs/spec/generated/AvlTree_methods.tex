
\subsubsection{\lst{AvlTree.digest} method (Code 100.1)}
\label{sec:type:AvlTree:digest}
\noindent
\begin{tabularx}{\textwidth}{| l | X |}
   \hline
   \bf{Description} & Returns digest of the state represented by this tree.
 Authenticated tree \lst{digest} = \lst{root hash bytes} ++ \lst{tree height}
         \\
   \hline
   \bf{Signature} & \lst{def digest}: \lst{Coll[Byte]} \\
  
  \hline
  
  \bf{Serialized as} & \hyperref[sec:serialization:operation:PropertyCall]{\lst{PropertyCall}} \\
  \hline
       
\end{tabularx}



\subsubsection{\lst{AvlTree.enabledOperations} method (Code 100.2)}
\label{sec:type:AvlTree:enabledOperations}
\noindent
\begin{tabularx}{\textwidth}{| l | X |}
   \hline
   \bf{Description} &  Flags of enabled operations packed in single byte.
 \lst{isInsertAllowed == (enabledOperations & 0x01) != 0}\newline
 \lst{isUpdateAllowed == (enabledOperations & 0x02) != 0}\newline
 \lst{isRemoveAllowed == (enabledOperations & 0x04) != 0}
         \\
   \hline
   \bf{Signature} & \lst{def enabledOperations}: \lst{Byte} \\
  
  \hline
  
  \bf{Serialized as} & \hyperref[sec:serialization:operation:PropertyCall]{\lst{PropertyCall}} \\
  \hline
       
\end{tabularx}



\subsubsection{\lst{AvlTree.keyLength} method (Code 100.3)}
\label{sec:type:AvlTree:keyLength}
\noindent
\begin{tabularx}{\textwidth}{| l | X |}
   \hline
   \bf{Description} & All the elements under the tree have the same given length of the keys. \\
   \hline
   \bf{Signature} & \lst{def keyLength}: \lst{Int} \\
  
  \hline
  
  \bf{Serialized as} & \hyperref[sec:serialization:operation:PropertyCall]{\lst{PropertyCall}} \\
  \hline
       
\end{tabularx}



\subsubsection{\lst{AvlTree.valueLengthOpt} method (Code 100.4)}
\label{sec:type:AvlTree:valueLengthOpt}
\noindent
\begin{tabularx}{\textwidth}{| l | X |}
   \hline
   \bf{Description} & If non-empty, all the values under the tree are of the same given length. \\
   \hline
   \bf{Signature} & \lst{def valueLengthOpt}: \lst{Option[Int]} \\
  
  \hline
  
  \bf{Serialized as} & \hyperref[sec:serialization:operation:PropertyCall]{\lst{PropertyCall}} \\
  \hline
       
\end{tabularx}



\subsubsection{\lst{AvlTree.isInsertAllowed} method (Code 100.5)}
\label{sec:type:AvlTree:isInsertAllowed}
\noindent
\begin{tabularx}{\textwidth}{| l | X |}
   \hline
   \bf{Description} & Checks if Insert operation is allowed for this tree instance. \\
   \hline
   \bf{Signature} & \lst{def isInsertAllowed}: \lst{Boolean} \\
  
  \hline
  
  \bf{Serialized as} & \hyperref[sec:serialization:operation:PropertyCall]{\lst{PropertyCall}} \\
  \hline
       
\end{tabularx}



\subsubsection{\lst{AvlTree.isUpdateAllowed} method (Code 100.6)}
\label{sec:type:AvlTree:isUpdateAllowed}
\noindent
\begin{tabularx}{\textwidth}{| l | X |}
   \hline
   \bf{Description} & Checks if Update operation is allowed for this tree instance. \\
   \hline
   \bf{Signature} & \lst{def isUpdateAllowed}: \lst{Boolean} \\
  
  \hline
  
  \bf{Serialized as} & \hyperref[sec:serialization:operation:PropertyCall]{\lst{PropertyCall}} \\
  \hline
       
\end{tabularx}



\subsubsection{\lst{AvlTree.isRemoveAllowed} method (Code 100.7)}
\label{sec:type:AvlTree:isRemoveAllowed}
\noindent
\begin{tabularx}{\textwidth}{| l | X |}
   \hline
   \bf{Description} & Checks if Remove operation is allowed for this tree instance. \\
   \hline
   \bf{Signature} & \lst{def isRemoveAllowed}: \lst{Boolean} \\
  
  \hline
  
  \bf{Serialized as} & \hyperref[sec:serialization:operation:PropertyCall]{\lst{PropertyCall}} \\
  \hline
       
\end{tabularx}



\subsubsection{\lst{AvlTree.updateOperations} method (Code 100.8)}
\label{sec:type:AvlTree:updateOperations}
\noindent
\begin{tabularx}{\textwidth}{| l | X |}
   \hline
   \bf{Description} &  Enable/disable operations of this tree producing a new tree.
 Since \lst{AvlTree} is immutable, \lst{this} tree instance remains unchanged.
         \\
   \hline
   \bf{Signature} & \lst{def updateOperations}(\lst{newOperations}$:$~\lst{Byte}): \lst{AvlTree} \\
  
  \hline
  \bf{Parameters} &
      \(\begin{array}{l l}
         \lst{newOperations} & \text{a new flags which specify available operations on a new tree} \\
      \end{array}\) \\
       
  \hline
  
  \bf{Serialized as} & \hyperref[sec:serialization:operation:MethodCall]{\lst{MethodCall}} \\
  \hline
       
\end{tabularx}



\subsubsection{\lst{AvlTree.contains} method (Code 100.9)}
\label{sec:type:AvlTree:contains}
\noindent
\begin{tabularx}{\textwidth}{| l | X |}
   \hline
   \bf{Description} &  Checks if an entry with key \lst{key} exists in this tree using proof \lst{proof}.

 NOTE, does not support multiple keys check, use \lst{getMany} instead.
 Returns \lst{true} if a leaf with the key \lst{key} exists.
 Returns \lst{false} if leaf with provided key does not exist.
         \\
   \hline
   \bf{Signature} & \lst{def contains}(\lst{key}$:$~\lst{Coll[Byte]}, \lst{proof}$:$~\lst{Coll[Byte]}): \lst{Boolean} \\
  
  \hline
  \bf{Parameters} &
      \(\begin{array}{l l}
         \lst{key} & \text{a key of an element of this authenticated dictionary} \\
\lst{proof} & \text{proof that they tree with \lst{this.digest} contains the given key} \\
      \end{array}\) \\
       
  \hline
  
  \bf{Serialized as} & \hyperref[sec:serialization:operation:MethodCall]{\lst{MethodCall}} \\
  \hline
       
\end{tabularx}



\subsubsection{\lst{AvlTree.get} method (Code 100.10)}
\label{sec:type:AvlTree:get}
\noindent
\begin{tabularx}{\textwidth}{| l | X |}
   \hline
   \bf{Description} &  Perform a lookup of key \lst{key} in this tree using \lst{proof}.
 Throws exception if proof is incorrect.

 NOTE, does not support multiple keys check, use \lst{getMany} instead.
 Return \lst{Some(bytes)} of leaf with key \lst{key} if it exists
 Return \lst{None} if leaf with provided key does not exist.
         \\
   \hline
   \bf{Signature} & \footnotesize \lst{def get}(\lst{key}$:$~\lst{Coll[Byte]}, \lst{proof}$:$~\lst{Coll[Byte]}): \lst{Option[Coll[Byte]]} \\
  
  \hline
  \bf{Parameters} &
      \(\begin{array}{l l}
         \lst{key} & \text{a key of an element of this authenticated dictionary} \\
\lst{proof} & \text{proof that they tree with \lst{this.digest} contains the given key} \\
      \end{array}\) \\
       
  \hline
  
  \bf{Serialized as} & \hyperref[sec:serialization:operation:MethodCall]{\lst{MethodCall}} \\
  \hline
       
\end{tabularx}



\subsubsection{\lst{AvlTree.getMany} method (Code 100.11)}
\label{sec:type:AvlTree:getMany}
\noindent
\begin{tabularx}{\textwidth}{| l | X |}
   \hline
   \bf{Description} &  Perform a lookup of many keys \lst{keys} in this tree using proof \lst{proof}.

 NOTE, keys must be ordered the same way they were in lookup before proof was generated.
 For each key return \lst{Some(bytes)} of leaf if it exists and \lst{None} if is doesn't.
         \\
   \hline
   \bf{Signature} & \footnotesize \lst{def getMany}(\lst{keys}$:$~\lst{Coll[Coll[Byte]]}, \lst{proof}$:$~\lst{Coll[Byte]}): \lst{Coll[Option[Coll[Byte]]]} \\
  
  \hline
  \bf{Parameters} &
      \(\begin{array}{l l}
         \lst{keys} & \text{keys of elements of this authenticated dictionary} \\
\lst{proof} & \text{proof that they tree with \lst{this.digest} contains the given key} \\
      \end{array}\) \\
       
  \hline
  
  \bf{Serialized as} & \hyperref[sec:serialization:operation:MethodCall]{\lst{MethodCall}} \\
  \hline
       
\end{tabularx}



\subsubsection{\lst{AvlTree.insert} method (Code 100.12)}
\label{sec:type:AvlTree:insert}
\noindent
\begin{tabularx}{\textwidth}{| l | X |}
   \hline
   \bf{Description} &  Perform insertions of key-value entries into this authenticated dictionary
 using proof \lst{proof}.
 Throws exception if proof is incorrect.

 NOTE, pairs must be ordered the same way they were in insert ops before proof was generated.
 Returns \lst{Some(newTree)} if successful.
 Returns \lst{None} if operations were not performed.
         \\
   \hline
   \bf{Signature} & \footnotesize \lst{def insert}(\lst{operations}$:$~\lst{Coll[(Coll[Byte],Coll[Byte])]}, \lst{proof}$:$~\lst{Coll[Byte]}): \lst{Option[AvlTree]} \\
  
  \hline
  \bf{Parameters} &
      \(\begin{array}{l l}
         \lst{operations} & \text{a collection of key-value pairs inserted in this dictionary} \\
\lst{proof} & \text{a proof that the key-value pairs were inserted} \\
      \end{array}\) \\
       
  \hline
  
  \bf{Serialized as} & \hyperref[sec:serialization:operation:MethodCall]{\lst{MethodCall}} \\
  \hline
       
\end{tabularx}



\subsubsection{\lst{AvlTree.update} method (Code 100.13)}
\label{sec:type:AvlTree:update}
\noindent
\begin{tabularx}{\textwidth}{| l | X |}
   \hline
   \bf{Description} &  Perform updates of key-value entries into this authenticated dictionary using proof \lst{proof}.
 Throws exception if proof is incorrect.

 Note, pairs must be ordered the same way they were in update ops before proof was generated.
 Returns \lst{Some(newTree)} if successful.
 Returns \lst{None} if operations were not performed.
         \\
   \hline
   \bf{Signature} & \footnotesize \lst{def update}(\lst{operations}$:$~\lst{Coll[(Coll[Byte],Coll[Byte])]}, \lst{proof}$:$~\lst{Coll[Byte]}): \lst{Option[AvlTree]} \\
  
  \hline
  \bf{Parameters} &
      \(\begin{array}{l l}
         \lst{operations} & \text{a collection of key-value pairs updated in this dictionary} \\
\lst{proof} & \text{a proof that the key-value pairs were updated} \\
      \end{array}\) \\
       
  \hline
  
  \bf{Serialized as} & \hyperref[sec:serialization:operation:MethodCall]{\lst{MethodCall}} \\
  \hline
       
\end{tabularx}



\subsubsection{\lst{AvlTree.remove} method (Code 100.14)}
\label{sec:type:AvlTree:remove}
\noindent
\begin{tabularx}{\textwidth}{| l | X |}
   \hline
   \bf{Description} &  Perform removal of entries into this authenticated dictionary using \lst{proof}.
 Throws exception if the proof is incorrect.
 Returns \lst{Some(newTree)} if successful.
 Returns \lst{None} if operations were not performed.
 NOTE, keys must be ordered the same way they were in remove ops before proof was generated.
         \\
   \hline
   \bf{Signature} & \footnotesize \lst{def remove}(\lst{operations}$:$~\lst{Coll[Coll[Byte]]}, \lst{proof}$:$~\lst{Coll[Byte]}): \lst{Option[AvlTree]} \\
  
  \hline
  \bf{Parameters} &
      \(\begin{array}{l l}
         \lst{operations} & \text{a collection of key-value pairs removed from this dictionary} \\
\lst{proof} & \text{a proof that the key-value pairs were removed} \\
      \end{array}\) \\
       
  \hline
  
  \bf{Serialized as} & \hyperref[sec:serialization:operation:MethodCall]{\lst{MethodCall}} \\
  \hline
       
\end{tabularx}



\subsubsection{\lst{AvlTree.updateDigest} method (Code 100.15)}
\label{sec:type:AvlTree:updateDigest}
\noindent
\begin{tabularx}{\textwidth}{| l | X |}
   \hline
   \bf{Description} &  Replace digest of this tree producing a new tree.
 Since AvlTree is immutable, this tree instance remains unchanged.
 Returns a copy of this AvlTree instance where \lst{this.digest} replaced by \lst{newDigest}.
         \\
   \hline
   \bf{Signature} & \lst{def updateDigest}(\lst{newDigest}$:$~\lst{Coll[Byte]}): \lst{AvlTree} \\
  
  \hline
  \bf{Parameters} &
      \(\begin{array}{l l}
         \lst{newDigest} & \text{a new digest} \\
      \end{array}\) \\
       
  \hline
  
  \bf{Serialized as} & \hyperref[sec:serialization:operation:MethodCall]{\lst{MethodCall}} \\
  \hline
       
\end{tabularx}
