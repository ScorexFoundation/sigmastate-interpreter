
\subsubsection{\lst{SOption.isEmpty} method (Code 36.1)}
\noindent
\begin{tabularx}{\textwidth}{| l | X |}
   \hline
   \bf{Description} &  \\
  
  \hline
  \bf{Parameters} &
      \(\begin{array}{l l l}
         \lst{arg0} & \lst{: Option[T]} & \text{// } \\
      \end{array}\) \\
       
  \hline
  \bf{Result} & \lst{Boolean} \\
  \hline
  
\end{tabularx}



\subsubsection{\lst{SOption.isDefined} method (Code 36.2)}
\noindent
\begin{tabularx}{\textwidth}{| l | X |}
   \hline
   \bf{Description} & Returns \lst{true} if the option is an instance of \lst{Some}, \lst{false} otherwise. \\
  
  \hline
  \bf{Parameters} &
      \(\begin{array}{l l l}
         
      \end{array}\) \\
       
  \hline
  \bf{Result} & \lst{Boolean} \\
  \hline
  
  \bf{Serialized as} & \lst{OptionIsDefined(opCode=230)} \\
  \hline
       
\end{tabularx}



\subsubsection{\lst{SOption.get} method (Code 36.3)}
\noindent
\begin{tabularx}{\textwidth}{| l | X |}
   \hline
   \bf{Description} & Returns the option's value. The option must be nonempty. Throws exception if the option is empty. \\
  
  \hline
  \bf{Parameters} &
      \(\begin{array}{l l l}
         
      \end{array}\) \\
       
  \hline
  \bf{Result} & \lst{T} \\
  \hline
  
  \bf{Serialized as} & \lst{OptionGet(opCode=228)} \\
  \hline
       
\end{tabularx}



\subsubsection{\lst{SOption.getOrElse} method (Code 36.4)}
\noindent
\begin{tabularx}{\textwidth}{| l | X |}
   \hline
   \bf{Description} & Returns the option's value if the option is nonempty, otherwise
return the result of evaluating \lst{default}.
         \\
  
  \hline
  \bf{Parameters} &
      \(\begin{array}{l l l}
         \lst{default} & \lst{: T} & \text{// the default value} \\
      \end{array}\) \\
       
  \hline
  \bf{Result} & \lst{T} \\
  \hline
  
  \bf{Serialized as} & \lst{OptionGetOrElse(opCode=229)} \\
  \hline
       
\end{tabularx}



\subsubsection{\lst{SOption.fold} method (Code 36.5)}
\noindent
\begin{tabularx}{\textwidth}{| l | X |}
   \hline
   \bf{Description} & Returns the result of applying \lst{f} to this option's
  value if the option is nonempty.  Otherwise, evaluates
  expression \lst{ifEmpty}.
  This is equivalent to \lst{option map f getOrElse ifEmpty}.
         \\
  
  \hline
  \bf{Parameters} &
      \(\begin{array}{l l l}
         \lst{ifEmpty} & \lst{: R} & \text{// the expression to evaluate if empty} \\
\lst{f} & \lst{: (T) => R} & \text{// the function to apply if nonempty} \\
      \end{array}\) \\
       
  \hline
  \bf{Result} & \lst{R} \\
  \hline
  
  \bf{Serialized as} & \lst{MethodCall(opCode=220)} \\
  \hline
       
\end{tabularx}



\subsubsection{\lst{SOption.toColl} method (Code 36.6)}
\noindent
\begin{tabularx}{\textwidth}{| l | X |}
   \hline
   \bf{Description} & Convert this Option to a collection with zero or one element. \\
  
  \hline
  \bf{Parameters} &
      \(\begin{array}{l l l}
         
      \end{array}\) \\
       
  \hline
  \bf{Result} & \lst{Coll[T]} \\
  \hline
  
  \bf{Serialized as} & \lst{PropertyCall(opCode=219)} \\
  \hline
       
\end{tabularx}



\subsubsection{\lst{SOption.map} method (Code 36.7)}
\noindent
\begin{tabularx}{\textwidth}{| l | X |}
   \hline
   \bf{Description} & Returns a \lst{Some} containing the result of applying \lst{f} to this option's
   value if this option is nonempty.
   Otherwise return \lst{None}.
         \\
  
  \hline
  \bf{Parameters} &
      \(\begin{array}{l l l}
         \lst{f} & \lst{: (T) => R} & \text{// the function to apply} \\
      \end{array}\) \\
       
  \hline
  \bf{Result} & \lst{Option[R]} \\
  \hline
  
  \bf{Serialized as} & \lst{MethodCall(opCode=220)} \\
  \hline
       
\end{tabularx}



\subsubsection{\lst{SOption.filter} method (Code 36.8)}
\noindent
\begin{tabularx}{\textwidth}{| l | X |}
   \hline
   \bf{Description} & Returns this option if it is nonempty and applying the predicate \lst{p} to
  this option's value returns true. Otherwise, return \lst{None}.
         \\
  
  \hline
  \bf{Parameters} &
      \(\begin{array}{l l l}
         \lst{p} & \lst{: (T) => Boolean} & \text{// the predicate used for testing} \\
      \end{array}\) \\
       
  \hline
  \bf{Result} & \lst{Option[T]} \\
  \hline
  
  \bf{Serialized as} & \lst{MethodCall(opCode=220)} \\
  \hline
       
\end{tabularx}



\subsubsection{\lst{SOption.flatMap} method (Code 36.9)}
\noindent
\begin{tabularx}{\textwidth}{| l | X |}
   \hline
   \bf{Description} & Returns the result of applying \lst{f} to this option's value if
   this option is nonempty.
   Returns \lst{None} if this option is empty.
   Slightly different from \lst{map} in that \lst{f} is expected to
   return an option (which could be \lst{one}).
         \\
  
  \hline
  \bf{Parameters} &
      \(\begin{array}{l l l}
         \lst{f} & \lst{: (T) => Option[R]} & \text{// the function to apply} \\
      \end{array}\) \\
       
  \hline
  \bf{Result} & \lst{Option[R]} \\
  \hline
  
  \bf{Serialized as} & \lst{MethodCall(opCode=220)} \\
  \hline
       
\end{tabularx}
