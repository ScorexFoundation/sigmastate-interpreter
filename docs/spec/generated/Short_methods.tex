
\subsubsection{\lst{Short.toByte} method (Code 106.1)}
\noindent
\begin{tabularx}{\textwidth}{| l | X |}
   \hline
   \bf{Description} & Converts this numeric value to \lst{Byte}, throwing exception if overflow. \\
  
  \hline
  \bf{Result} & \lst{Byte} \\
  \hline
  
\end{tabularx}



\subsubsection{\lst{Short.toShort} method (Code 106.2)}
\noindent
\begin{tabularx}{\textwidth}{| l | X |}
   \hline
   \bf{Description} & Converts this numeric value to \lst{Short}, throwing exception if overflow. \\
  
  \hline
  \bf{Result} & \lst{Short} \\
  \hline
  
\end{tabularx}



\subsubsection{\lst{Short.toInt} method (Code 106.3)}
\noindent
\begin{tabularx}{\textwidth}{| l | X |}
   \hline
   \bf{Description} & Converts this numeric value to \lst{Int}, throwing exception if overflow. \\
  
  \hline
  \bf{Result} & \lst{Int} \\
  \hline
  
\end{tabularx}



\subsubsection{\lst{Short.toLong} method (Code 106.4)}
\noindent
\begin{tabularx}{\textwidth}{| l | X |}
   \hline
   \bf{Description} & Converts this numeric value to \lst{Long}, throwing exception if overflow. \\
  
  \hline
  \bf{Result} & \lst{Long} \\
  \hline
  
\end{tabularx}



\subsubsection{\lst{Short.toBigInt} method (Code 106.5)}
\noindent
\begin{tabularx}{\textwidth}{| l | X |}
   \hline
   \bf{Description} & Converts this numeric value to \lst{BigInt} \\
  
  \hline
  \bf{Result} & \lst{BigInt} \\
  \hline
  
\end{tabularx}



\subsubsection{\lst{Short.toBytes} method (Code 106.6)}
\noindent
\begin{tabularx}{\textwidth}{| l | X |}
   \hline
   \bf{Description} & Returns a big-endian representation of this numeric value in a collection of bytes.
 For example, the Int value \lst{0x12131415} would yield the
 byte array  \lst{[0x12, 0x13, 0x14, 0x15]}. \\
  
  \hline
  \bf{Result} & \lst{Coll[Byte]} \\
  \hline
  
  \bf{Serialized as} & \lst{PropertyCall(opCode=219)} \\
  \hline
       
\end{tabularx}



\subsubsection{\lst{Short.toBits} method (Code 106.7)}
\noindent
\begin{tabularx}{\textwidth}{| l | X |}
   \hline
   \bf{Description} & Returns a big-endian representation of this numeric in a collection of Booleans.
 Each boolean corresponds to one bit. \\
  
  \hline
  \bf{Result} & \lst{Coll[Boolean]} \\
  \hline
  
  \bf{Serialized as} & \lst{PropertyCall(opCode=219)} \\
  \hline
       
\end{tabularx}
