\section{Predefined types}
\label{sec:appendix:predeftypes}

\begin{table}[h]
    \small
    \begin{tabu}{|l |l |l |l |l |l |l |l|}
     \hline
     \rowfont{\bfseries}
        Name   &   Code   &  IsConstSize & 
        isPrim\footnote{isPrim - primitive type} & 
        isEmbed  & isNum & Set of values \\
        \hline

\begin{tabu}{|l |l |l |l |l |l |l |l|}
    \hline
    \rowfont{\bfseries}
       Name   &   Code   &  IsConstSize & 
       isPrim\footnote{isPrim - primitive type} & 
       isEmbed  & isNum & Set of values \\
\hline
\lst{Boolean}	&	$1$	&	\lst{true}	& \lst{true}	&	\lst{true} &	\lst{false}	& $\Set{\lst{true}, \lst{false}}$ \\
\hline
\lst{Byte}	&	$2$	&	\lst{true}	& \lst{true}	&	\lst{true} &	\lst{true}	& $\Set{-2^{7} \dots 2^{7}-1}$ \\
\hline
\lst{Short}	&	$3$	&	\lst{true}	& \lst{true}	&	\lst{true} &	\lst{true}	& $\Set{-2^{15} \dots 2^{15}-1}$ \\
\hline
\lst{Int}	&	$4$	&	\lst{true}	& \lst{true}	&	\lst{true} &	\lst{true}	& $\Set{-2^{31} \dots 2^{31}-1}$ \\
\hline
\lst{Long}	&	$5$	&	\lst{true}	& \lst{true}	&	\lst{true} &	\lst{true}	& $\Set{-2^{63} \dots 2^{63}-1}$ \\
\hline
\lst{BigInt}	&	$6$	&	\lst{true}	& \lst{true}	&	\lst{true} &	\lst{true}	& $\Set{-2^{255} \dots 2^{255}-1}$ \\
\hline
\lst{GroupElement}	&	$7$	&	\lst{true}	& \lst{true}	&	\lst{true} &	\lst{false}	& $\Set{p \in \lst{SecP256K1Point}}$ \\
\hline
\lst{SigmaProp}	&	$8$	&	\lst{false}	& \lst{true}	&	\lst{true} &	\lst{false}	& Sec.~\ref{sec:type:SigmaProp} \\
\hline
\lst{Box}	&	$99$	&	\lst{false}	& \lst{false}	&	\lst{false} &	\lst{false}	& Sec.~\ref{sec:type:Box} \\
\hline
\lst{AvlTree}	&	$100$	&	\lst{false}	& \lst{false}	&	\lst{false} &	\lst{false}	& Sec.~\ref{sec:type:AvlTree} \\
\hline
\lst{Context}	&	$101$	&	\lst{false}	& \lst{false}	&	\lst{false} &	\lst{false}	& Sec.~\ref{sec:type:Context} \\
\hline
\lst{Header}	&	$104$	&	\lst{true}	& \lst{false}	&	\lst{false} &	\lst{false}	& Sec.~\ref{sec:type:Header} \\
\hline
\lst{PreHeader}	&	$105$	&	\lst{true}	& \lst{false}	&	\lst{false} &	\lst{false}	& Sec.~\ref{sec:type:PreHeader} \\
\hline
\lst{SigmaDslBuilder}	&	$106$	&	\lst{true}	& \lst{false}	&	\lst{false} &	\lst{false}	& Sec.~\ref{sec:type:SigmaDslBuilder} \\
\hline
\end{tabu}

    \hline
    \end{tabu}
    \caption{Predefined types of \langname}
    \label{table:predeftypes}
\end{table}

\mnote{This table should be autogenerated from sigma SType descriptors}

\subsection{SigmaProp type}
\label{sec:type:SigmaProp}

Values of \lst{SigmaProp} type hold sigma propositions, which can be proved
and verified using Sigma protocols. Each sigma proposition is represented as
an expression where sigma protocol primitives such as \lst{ProveDlog}, and
\lst{ProveDHTuple} are used as constants and special sigma protocol
connectives like \lst{&&},\lst{||} and \lst{THRESHOLD} are used as operations.

The abstract syntax of sigma propositions is shown in
Figure~\ref{fig:sigmaprop:tree}.

\begin{figure}[h]
   \centering
   \begin{tabular}{@{}l c l l l} 
      \hline
      Set 		&  			& Syntax	   & Mnemonic 	& Description \\
      \hline
      $Tree \ni t$	& := 	& \lst{Trivial(b)} 	& \lst{TrivialProp}	& boolean value \lst{b} as sigma proposition  \\
                     & $\mid$	& \lst{Dlog(ge)} 	& \lst{ProveDLog}	& knowledge of discrete logarithm of \lst{ge} \\
                     & $\mid$ & \lst{DHTuple(g,h,u,v)} 	& \lst{ProveDHTuple}	& knowledge of Diffie-Hellman tuple \\
                     & $\mid$ & \lst{THRESHOLD}$(k,t_1,\dots,t_n)$ 	& \lst{THRESHOLD}	& knowledge of $k$ out of $n$ secrets\\
                     & $\mid$ & \lst{OR}$(t_1,\dots,t_n)$ 	& \lst{OR}	& knowledge of any one of $n$ secrets\\
                     & $\mid$ & \lst{AND}$(t_1,\dots,t_n)$ 	& \lst{AND}	& knowledge of all $n$ secrets\\
      \end{tabular} 
   \caption{Abstract syntax of sigma propositions}
   \label{fig:sigmaprop:tree}
\end{figure}

Every well-formed tree of sigma proposition is a value of type
\lst{SigmaProp}, thus following the notation of Section~\ref{sec:evaluation}
we can define denotation of \lst{SigmaProp}
$$\Denot{\lst{SigmaProp}} = \Set{t \in Tree}$$


The following methods can be called on all instances of \lst{SigmaProp} type.

\noindent
\begin{tabularx}{\textwidth}{| l | X |}
  \hline
  \bf{Method Signature} & \bf{Description} \\
  \hline
  \lst{def propBytes: Coll[Byte]} & 
    Serialized bytes of this sigma proposition taken as ErgoTree and then serialized. \\
  \hline
\end{tabularx}

For a list of primitive operations on \lst{SigmaProp} type see Appendix~\ref{sec:appendix:primops}.

\subsection{Box type}
\label{sec:type:Box}

\subsection{AvlTree type}
\label{sec:type:AvlTree}

\subsection{Context type}
\label{sec:type:Context}

\subsection{Header type}
\label{sec:type:Header}

\subsection{PreHeader type}
\label{sec:type:PreHeader}

\subsection{Global type}
\label{sec:type:SigmaDslBuilder}