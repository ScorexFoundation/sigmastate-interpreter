\section{Predefined types}
\label{sec:appendix:predeftypes}

\begin{table}[h]
    \small
    \begin{tabu}{|l |l |l |l |l |l |l |l|}
     \hline
     \rowfont{\bfseries}
        Name   &   Code   &  IsConstSize & 
        isPrim\footnote{isPrim - primitive type} & 
        isEmbed  & isNum & Set of values \\
        \hline

\lst{Boolean}	&	$1$	&	\lst{isConst}	& \lst{true}	&	\lst{true} &	\lst{false}	& $\Set{\lst{true}, \lst{false}}$ \\
\hline
\lst{Byte}	&	$2$	&	\lst{isConst}	& \lst{true}	&	\lst{true} &	\lst{true}	& $\Set{-2^{7} \dots 2^{7}-1}$~\ref{sec:type:Byte} \\
\hline
\lst{Short}	&	$3$	&	\lst{isConst}	& \lst{true}	&	\lst{true} &	\lst{true}	& $\Set{-2^{15} \dots 2^{15}-1}$~\ref{sec:type:Short} \\
\hline
\lst{Int}	&	$4$	&	\lst{isConst}	& \lst{true}	&	\lst{true} &	\lst{true}	& $\Set{-2^{31} \dots 2^{31}-1}$~\ref{sec:type:Int} \\
\hline
\lst{Long}	&	$5$	&	\lst{isConst}	& \lst{true}	&	\lst{true} &	\lst{true}	& $\Set{-2^{63} \dots 2^{63}-1}$~\ref{sec:type:Long} \\
\hline
\lst{BigInt}	&	$6$	&	\lst{isConst}	& \lst{true}	&	\lst{true} &	\lst{true}	& $\Set{-2^{255} \dots 2^{255}-1}$~\ref{sec:type:BigInt} \\
\hline
\lst{GroupElement}	&	$7$	&	\lst{isConst}	& \lst{true}	&	\lst{true} &	\lst{false}	& $\Set{p \in \lst{SecP256K1Point}}$ \\
\hline
\lst{SigmaProp}	&	$8$	&	\lst{isConst}	& \lst{true}	&	\lst{true} &	\lst{false}	& Sec.~\ref{sec:type:SigmaProp} \\
\hline
\lst{Any}	&	$97$	&	\lst{isConst}	& \lst{true}	&	\lst{false} &	\lst{false}	& Sec.~\ref{sec:type:Any} \\
\hline
\lst{Unit}	&	$98$	&	\lst{isConst}	& \lst{true}	&	\lst{false} &	\lst{false}	& Sec.~\ref{sec:type:Unit} \\
\hline
\lst{Box}	&	$99$	&	\lst{isConst}	& \lst{false}	&	\lst{false} &	\lst{false}	& Sec.~\ref{sec:type:Box} \\
\hline
\lst{AvlTree}	&	$100$	&	\lst{isConst}	& \lst{false}	&	\lst{false} &	\lst{false}	& Sec.~\ref{sec:type:AvlTree} \\
\hline
\lst{Context}	&	$101$	&	\lst{isConst}	& \lst{false}	&	\lst{false} &	\lst{false}	& Sec.~\ref{sec:type:Context} \\
\hline
\lst{Header}	&	$104$	&	\lst{isConst}	& \lst{false}	&	\lst{false} &	\lst{false}	& Sec.~\ref{sec:type:Header} \\
\hline
\lst{PreHeader}	&	$105$	&	\lst{isConst}	& \lst{false}	&	\lst{false} &	\lst{false}	& Sec.~\ref{sec:type:PreHeader} \\
\hline
\lst{Global}	&	$106$	&	\lst{isConst}	& \lst{false}	&	\lst{false} &	\lst{false}	& Sec.~\ref{sec:type:Global} \\

    \hline
    \end{tabu}
    \caption{Predefined types of \langname}
    \label{table:predeftypes}
\end{table}

There is a section for each type with sub-sections for all available methods. Each
method is characterized by the description, signature (i.e. name, parameters and return
type), description of all parameters and reference to the serialization format.

There is universal primitive which can represent any method invocation
(\lst{MethodCall}). However, many method are also mapped to the special primitive
operations to save storage space.

The following sub-sections are auto-generated from type descriptors of \langname reference
implementation.

% \subsection{Boolean type}
% \label{sec:type:Boolean}
% 
\subsubsection{\lst{Boolean.toByte} method (Code 1.1)}
\label{sec:type:Boolean:toByte}
\noindent
\begin{tabularx}{\textwidth}{| l | X |}
   \hline
   \bf{Description} & Convert true to 1 and false to 0 \\
  
  \hline
  \bf{Parameters} &
      \(\begin{array}{l l l}
         
      \end{array}\) \\
       
  \hline
  \bf{Result} & \lst{Byte} \\
  \hline
  
  \bf{Serialized as} & \hyperref[sec:serialization:operation:PropertyCall]{\lst{PropertyCall}} \\
  \hline
       
\end{tabularx}


\subsection{Byte type}
\label{sec:type:Byte}

\subsubsection{\lst{Byte.toByte} method (Code 106.1)}
\label{sec:type:Byte:toByte}
\noindent
\begin{tabularx}{\textwidth}{| l | X |}
   \hline
   \bf{Description} & Converts this numeric value to \lst{Byte}, throwing exception if overflow. \\
   \hline
   \bf{Signature} & \lst{def toByte}: \lst{Byte} \\
  
  \hline
  
  \bf{Serialized as} & \hyperref[sec:serialization:operation:Downcast]{\lst{Downcast}} \\
  \hline
       
\end{tabularx}



\subsubsection{\lst{Byte.toShort} method (Code 106.2)}
\label{sec:type:Byte:toShort}
\noindent
\begin{tabularx}{\textwidth}{| l | X |}
   \hline
   \bf{Description} & Converts this numeric value to \lst{Short}, throwing exception if overflow. \\
   \hline
   \bf{Signature} & \lst{def toShort}: \lst{Short} \\
  
  \hline
  
  \bf{Serialized as} & \hyperref[sec:serialization:operation:Upcast]{\lst{Upcast}} \\
  \hline
       
\end{tabularx}



\subsubsection{\lst{Byte.toInt} method (Code 106.3)}
\label{sec:type:Byte:toInt}
\noindent
\begin{tabularx}{\textwidth}{| l | X |}
   \hline
   \bf{Description} & Converts this numeric value to \lst{Int}, throwing exception if overflow. \\
   \hline
   \bf{Signature} & \lst{def toInt}: \lst{Int} \\
  
  \hline
  
  \bf{Serialized as} & \hyperref[sec:serialization:operation:Upcast]{\lst{Upcast}} \\
  \hline
       
\end{tabularx}



\subsubsection{\lst{Byte.toLong} method (Code 106.4)}
\label{sec:type:Byte:toLong}
\noindent
\begin{tabularx}{\textwidth}{| l | X |}
   \hline
   \bf{Description} & Converts this numeric value to \lst{Long}, throwing exception if overflow. \\
   \hline
   \bf{Signature} & \lst{def toLong}: \lst{Long} \\
  
  \hline
  
  \bf{Serialized as} & \hyperref[sec:serialization:operation:Upcast]{\lst{Upcast}} \\
  \hline
       
\end{tabularx}



\subsubsection{\lst{Byte.toBigInt} method (Code 106.5)}
\label{sec:type:Byte:toBigInt}
\noindent
\begin{tabularx}{\textwidth}{| l | X |}
   \hline
   \bf{Description} & Converts this numeric value to \lst{BigInt} \\
   \hline
   \bf{Signature} & \lst{def toBigInt}: \lst{BigInt} \\
  
  \hline
  
  \bf{Serialized as} & \hyperref[sec:serialization:operation:Upcast]{\lst{Upcast}} \\
  \hline
       
\end{tabularx}


\subsection{Short type}
\label{sec:type:Short}
 
\subsubsection{\lst{Short.toByte} method (Code 106.1)}
\label{sec:type:Short:toByte}
\noindent
\begin{tabularx}{\textwidth}{| l | X |}
   \hline
   \bf{Description} & Converts this numeric value to \lst{Byte}, throwing exception if overflow. \\
   \hline
   \bf{Signature} & \lst{def toByte}: \lst{Byte} \\
  
  \hline
  
  \bf{Serialized as} & \hyperref[sec:serialization:operation:Downcast]{\lst{Downcast}} \\
  \hline
       
\end{tabularx}



\subsubsection{\lst{Short.toShort} method (Code 106.2)}
\label{sec:type:Short:toShort}
\noindent
\begin{tabularx}{\textwidth}{| l | X |}
   \hline
   \bf{Description} & Converts this numeric value to \lst{Short}, throwing exception if overflow. \\
   \hline
   \bf{Signature} & \lst{def toShort}: \lst{Short} \\
  
  \hline
  
  \bf{Serialized as} & \hyperref[sec:serialization:operation:Downcast]{\lst{Downcast}} \\
  \hline
       
\end{tabularx}



\subsubsection{\lst{Short.toInt} method (Code 106.3)}
\label{sec:type:Short:toInt}
\noindent
\begin{tabularx}{\textwidth}{| l | X |}
   \hline
   \bf{Description} & Converts this numeric value to \lst{Int}, throwing exception if overflow. \\
   \hline
   \bf{Signature} & \lst{def toInt}: \lst{Int} \\
  
  \hline
  
  \bf{Serialized as} & \hyperref[sec:serialization:operation:Upcast]{\lst{Upcast}} \\
  \hline
       
\end{tabularx}



\subsubsection{\lst{Short.toLong} method (Code 106.4)}
\label{sec:type:Short:toLong}
\noindent
\begin{tabularx}{\textwidth}{| l | X |}
   \hline
   \bf{Description} & Converts this numeric value to \lst{Long}, throwing exception if overflow. \\
   \hline
   \bf{Signature} & \lst{def toLong}: \lst{Long} \\
  
  \hline
  
  \bf{Serialized as} & \hyperref[sec:serialization:operation:Upcast]{\lst{Upcast}} \\
  \hline
       
\end{tabularx}



\subsubsection{\lst{Short.toBigInt} method (Code 106.5)}
\label{sec:type:Short:toBigInt}
\noindent
\begin{tabularx}{\textwidth}{| l | X |}
   \hline
   \bf{Description} & Converts this numeric value to \lst{BigInt} \\
   \hline
   \bf{Signature} & \lst{def toBigInt}: \lst{BigInt} \\
  
  \hline
  
  \bf{Serialized as} & \hyperref[sec:serialization:operation:Upcast]{\lst{Upcast}} \\
  \hline
       
\end{tabularx}


\subsection{Int type}
\label{sec:type:Int}
 
\subsubsection{\lst{Int.toByte} method (Code 106.1)}
\label{sec:type:Int:toByte}
\noindent
\begin{tabularx}{\textwidth}{| l | X |}
   \hline
   \bf{Description} & Converts this numeric value to \lst{Byte}, throwing exception if overflow. \\
   \hline
   \bf{Signature} & \lst{def toByte}: \lst{Byte} \\
  
  \hline
  
  \bf{Serialized as} & \hyperref[sec:serialization:operation:Downcast]{\lst{Downcast}} \\
  \hline
       
\end{tabularx}



\subsubsection{\lst{Int.toShort} method (Code 106.2)}
\label{sec:type:Int:toShort}
\noindent
\begin{tabularx}{\textwidth}{| l | X |}
   \hline
   \bf{Description} & Converts this numeric value to \lst{Short}, throwing exception if overflow. \\
   \hline
   \bf{Signature} & \lst{def toShort}: \lst{Short} \\
  
  \hline
  
  \bf{Serialized as} & \hyperref[sec:serialization:operation:Downcast]{\lst{Downcast}} \\
  \hline
       
\end{tabularx}



\subsubsection{\lst{Int.toInt} method (Code 106.3)}
\label{sec:type:Int:toInt}
\noindent
\begin{tabularx}{\textwidth}{| l | X |}
   \hline
   \bf{Description} & Converts this numeric value to \lst{Int}, throwing exception if overflow. \\
   \hline
   \bf{Signature} & \lst{def toInt}: \lst{Int} \\
  
  \hline
  
  \bf{Serialized as} & \hyperref[sec:serialization:operation:Downcast]{\lst{Downcast}} \\
  \hline
       
\end{tabularx}



\subsubsection{\lst{Int.toLong} method (Code 106.4)}
\label{sec:type:Int:toLong}
\noindent
\begin{tabularx}{\textwidth}{| l | X |}
   \hline
   \bf{Description} & Converts this numeric value to \lst{Long}, throwing exception if overflow. \\
   \hline
   \bf{Signature} & \lst{def toLong}: \lst{Long} \\
  
  \hline
  
  \bf{Serialized as} & \hyperref[sec:serialization:operation:Upcast]{\lst{Upcast}} \\
  \hline
       
\end{tabularx}



\subsubsection{\lst{Int.toBigInt} method (Code 106.5)}
\label{sec:type:Int:toBigInt}
\noindent
\begin{tabularx}{\textwidth}{| l | X |}
   \hline
   \bf{Description} & Converts this numeric value to \lst{BigInt} \\
   \hline
   \bf{Signature} & \lst{def toBigInt}: \lst{BigInt} \\
  
  \hline
  
  \bf{Serialized as} & \hyperref[sec:serialization:operation:Upcast]{\lst{Upcast}} \\
  \hline
       
\end{tabularx}


\subsection{Long type}
\label{sec:type:Long}
 
\subsubsection{\lst{Long.toByte} method (Code 106.1)}
\label{sec:type:Long:toByte}
\noindent
\begin{tabularx}{\textwidth}{| l | X |}
   \hline
   \bf{Description} & Converts this numeric value to \lst{Byte}, throwing exception if overflow. \\
   \hline
   \bf{Signature} & \lst{def toByte}: \lst{Byte} \\
  
  \hline
  
  \bf{Serialized as} & \hyperref[sec:serialization:operation:Downcast]{\lst{Downcast}} \\
  \hline
       
\end{tabularx}



\subsubsection{\lst{Long.toShort} method (Code 106.2)}
\label{sec:type:Long:toShort}
\noindent
\begin{tabularx}{\textwidth}{| l | X |}
   \hline
   \bf{Description} & Converts this numeric value to \lst{Short}, throwing exception if overflow. \\
   \hline
   \bf{Signature} & \lst{def toShort}: \lst{Short} \\
  
  \hline
  
  \bf{Serialized as} & \hyperref[sec:serialization:operation:Downcast]{\lst{Downcast}} \\
  \hline
       
\end{tabularx}



\subsubsection{\lst{Long.toInt} method (Code 106.3)}
\label{sec:type:Long:toInt}
\noindent
\begin{tabularx}{\textwidth}{| l | X |}
   \hline
   \bf{Description} & Converts this numeric value to \lst{Int}, throwing exception if overflow. \\
   \hline
   \bf{Signature} & \lst{def toInt}: \lst{Int} \\
  
  \hline
  
  \bf{Serialized as} & \hyperref[sec:serialization:operation:Downcast]{\lst{Downcast}} \\
  \hline
       
\end{tabularx}



\subsubsection{\lst{Long.toLong} method (Code 106.4)}
\label{sec:type:Long:toLong}
\noindent
\begin{tabularx}{\textwidth}{| l | X |}
   \hline
   \bf{Description} & Converts this numeric value to \lst{Long}, throwing exception if overflow. \\
   \hline
   \bf{Signature} & \lst{def toLong}: \lst{Long} \\
  
  \hline
  
  \bf{Serialized as} & \hyperref[sec:serialization:operation:Downcast]{\lst{Downcast}} \\
  \hline
       
\end{tabularx}



\subsubsection{\lst{Long.toBigInt} method (Code 106.5)}
\label{sec:type:Long:toBigInt}
\noindent
\begin{tabularx}{\textwidth}{| l | X |}
   \hline
   \bf{Description} & Converts this numeric value to \lst{BigInt} \\
   \hline
   \bf{Signature} & \lst{def toBigInt}: \lst{BigInt} \\
  
  \hline
  
  \bf{Serialized as} & \hyperref[sec:serialization:operation:Upcast]{\lst{Upcast}} \\
  \hline
       
\end{tabularx}


\subsection{BigInt type}
\label{sec:type:BigInt}
 
\subsubsection{\lst{BigInt.toByte} method (Code 106.1)}
\label{sec:type:BigInt:toByte}
\noindent
\begin{tabularx}{\textwidth}{| l | X |}
   \hline
   \bf{Description} & Converts this numeric value to \lst{Byte}, throwing exception if overflow. \\
  
  \hline
  \bf{Parameters} &
      \(\begin{array}{l l l}
         
      \end{array}\) \\
       
  \hline
  \bf{Result} & \lst{Byte} \\
  \hline
  
  \bf{Serialized as} & \hyperref[sec:serialization:operation:PropertyCall]{\lst{PropertyCall}} \\
  \hline
       
\end{tabularx}



\subsubsection{\lst{BigInt.toShort} method (Code 106.2)}
\label{sec:type:BigInt:toShort}
\noindent
\begin{tabularx}{\textwidth}{| l | X |}
   \hline
   \bf{Description} & Converts this numeric value to \lst{Short}, throwing exception if overflow. \\
  
  \hline
  \bf{Parameters} &
      \(\begin{array}{l l l}
         
      \end{array}\) \\
       
  \hline
  \bf{Result} & \lst{Short} \\
  \hline
  
  \bf{Serialized as} & \hyperref[sec:serialization:operation:PropertyCall]{\lst{PropertyCall}} \\
  \hline
       
\end{tabularx}



\subsubsection{\lst{BigInt.toInt} method (Code 106.3)}
\label{sec:type:BigInt:toInt}
\noindent
\begin{tabularx}{\textwidth}{| l | X |}
   \hline
   \bf{Description} & Converts this numeric value to \lst{Int}, throwing exception if overflow. \\
  
  \hline
  \bf{Parameters} &
      \(\begin{array}{l l l}
         
      \end{array}\) \\
       
  \hline
  \bf{Result} & \lst{Int} \\
  \hline
  
  \bf{Serialized as} & \hyperref[sec:serialization:operation:PropertyCall]{\lst{PropertyCall}} \\
  \hline
       
\end{tabularx}



\subsubsection{\lst{BigInt.toLong} method (Code 106.4)}
\label{sec:type:BigInt:toLong}
\noindent
\begin{tabularx}{\textwidth}{| l | X |}
   \hline
   \bf{Description} & Converts this numeric value to \lst{Long}, throwing exception if overflow. \\
  
  \hline
  \bf{Parameters} &
      \(\begin{array}{l l l}
         
      \end{array}\) \\
       
  \hline
  \bf{Result} & \lst{Long} \\
  \hline
  
  \bf{Serialized as} & \hyperref[sec:serialization:operation:PropertyCall]{\lst{PropertyCall}} \\
  \hline
       
\end{tabularx}



\subsubsection{\lst{BigInt.toBigInt} method (Code 106.5)}
\label{sec:type:BigInt:toBigInt}
\noindent
\begin{tabularx}{\textwidth}{| l | X |}
   \hline
   \bf{Description} & Converts this numeric value to \lst{BigInt} \\
  
  \hline
  \bf{Parameters} &
      \(\begin{array}{l l l}
         
      \end{array}\) \\
       
  \hline
  \bf{Result} & \lst{BigInt} \\
  \hline
  
  \bf{Serialized as} & \hyperref[sec:serialization:operation:PropertyCall]{\lst{PropertyCall}} \\
  \hline
       
\end{tabularx}



\subsubsection{\lst{BigInt.toBytes} method (Code 106.6)}
\label{sec:type:BigInt:toBytes}
\noindent
\begin{tabularx}{\textwidth}{| l | X |}
   \hline
   \bf{Description} &  Returns a big-endian representation of this numeric value in a collection of bytes.
 For example, the \lst{Int} value \lst{0x12131415} would yield the
 collection of bytes \lst{[0x12, 0x13, 0x14, 0x15]}.
           \\
  
  \hline
  \bf{Parameters} &
      \(\begin{array}{l l l}
         
      \end{array}\) \\
       
  \hline
  \bf{Result} & \lst{Coll[Byte]} \\
  \hline
  
  \bf{Serialized as} & \hyperref[sec:serialization:operation:PropertyCall]{\lst{PropertyCall}} \\
  \hline
       
\end{tabularx}



\subsubsection{\lst{BigInt.toBits} method (Code 106.7)}
\label{sec:type:BigInt:toBits}
\noindent
\begin{tabularx}{\textwidth}{| l | X |}
   \hline
   \bf{Description} &  Returns a big-endian representation of this numeric in a collection of Booleans.
  Each boolean corresponds to one bit.
           \\
  
  \hline
  \bf{Parameters} &
      \(\begin{array}{l l l}
         
      \end{array}\) \\
       
  \hline
  \bf{Result} & \lst{Coll[Boolean]} \\
  \hline
  
  \bf{Serialized as} & \hyperref[sec:serialization:operation:PropertyCall]{\lst{PropertyCall}} \\
  \hline
       
\end{tabularx}


\subsection{GroupElement type}
\label{sec:type:GroupElement}

\subsubsection{\lst{GroupElement.getEncoded} method (Code 7.2)}
\label{sec:type:GroupElement:getEncoded}
\noindent
\begin{tabularx}{\textwidth}{| l | X |}
   \hline
   \bf{Description} & Get an encoding of the point value. \\
  
  \hline
  \bf{Parameters} &
      \(\begin{array}{l l l}
         
      \end{array}\) \\
       
  \hline
  \bf{Result} & \lst{Coll[Byte]} \\
  \hline
  
  \bf{Serialized as} & \hyperref[sec:serialization:operation:PropertyCall]{\lst{PropertyCall}} \\
  \hline
       
\end{tabularx}



\subsubsection{\lst{GroupElement.exp} method (Code 7.3)}
\label{sec:type:GroupElement:exp}
\noindent
\begin{tabularx}{\textwidth}{| l | X |}
   \hline
   \bf{Description} & Exponentiate this \lst{GroupElement} to the given number. Returns this to the power of k \\
  
  \hline
  \bf{Parameters} &
      \(\begin{array}{l l l}
         \lst{k} & \lst{: BigInt} & \text{// The power} \\
      \end{array}\) \\
       
  \hline
  \bf{Result} & \lst{GroupElement} \\
  \hline
  
  \bf{Serialized as} & \hyperref[sec:serialization:operation:Exponentiate]{\lst{Exponentiate}} \\
  \hline
       
\end{tabularx}



\subsubsection{\lst{GroupElement.multiply} method (Code 7.4)}
\label{sec:type:GroupElement:multiply}
\noindent
\begin{tabularx}{\textwidth}{| l | X |}
   \hline
   \bf{Description} & Group operation. \\
  
  \hline
  \bf{Parameters} &
      \(\begin{array}{l l l}
         \lst{other} & \lst{: GroupElement} & \text{// other element of the group} \\
      \end{array}\) \\
       
  \hline
  \bf{Result} & \lst{GroupElement} \\
  \hline
  
  \bf{Serialized as} & \hyperref[sec:serialization:operation:MultiplyGroup]{\lst{MultiplyGroup}} \\
  \hline
       
\end{tabularx}



\subsubsection{\lst{GroupElement.negate} method (Code 7.5)}
\label{sec:type:GroupElement:negate}
\noindent
\begin{tabularx}{\textwidth}{| l | X |}
   \hline
   \bf{Description} & Inverse element of the group. \\
  
  \hline
  \bf{Parameters} &
      \(\begin{array}{l l l}
         
      \end{array}\) \\
       
  \hline
  \bf{Result} & \lst{GroupElement} \\
  \hline
  
  \bf{Serialized as} & \hyperref[sec:serialization:operation:PropertyCall]{\lst{PropertyCall}} \\
  \hline
       
\end{tabularx}


\subsection{SigmaProp type}
\label{sec:type:SigmaProp}

Values of \lst{SigmaProp} type hold sigma propositions, which can be proved
and verified using Sigma protocols. Each sigma proposition is represented as
an expression where sigma protocol primitives such as \lst{ProveDlog}, and
\lst{ProveDHTuple} are used as constants and special sigma protocol
connectives like \lst{&&},\lst{||} and \lst{THRESHOLD} are used as operations.

The abstract syntax of sigma propositions is shown in
Figure~\ref{fig:sigmaprop:tree}.

\begin{figure}[h] \footnotesize
   \caption{Abstract syntax of sigma propositions}\vspace{-7pt}
   \label{fig:sigmaprop:tree}
   \centering
   \begin{tabular}{@{}l c l l l} 
      \hline
      Set 		&  			& Syntax	   & Mnemonic 	& Description \\
      \hline
      $Tree \ni t$	& := 	& \lst{Trivial(b)} 	& \lst{TrivialProp}	& boolean value \lst{b} as sigma proposition  \\
                     & $\mid$	& \lst{Dlog(ge)} 	& \lst{ProveDLog}	& knowledge of discrete logarithm of \lst{ge} \\
                     & $\mid$ & \lst{DHTuple(g,h,u,v)} 	& \lst{ProveDHTuple}	& knowledge of Diffie-Hellman tuple \\
                     & $\mid$ & \lst{THRESHOLD}$(k,t_1,\dots,t_n)$ 	& \lst{CTHRESHOLD}	& knowledge of $k$ out of $n$ secrets\\
                     & $\mid$ & \lst{OR}$(t_1,\dots,t_n)$ 	& \lst{COR}	& knowledge of any one of $n$ secrets\\
                     & $\mid$ & \lst{AND}$(t_1,\dots,t_n)$ 	& \lst{CAND}	& knowledge of all $n$ secrets\\
      \hline
   \end{tabular} 
\end{figure}

Every well-formed tree of sigma proposition is a value of type
\lst{SigmaProp}, thus following the notation of Section~\ref{sec:evaluation}
we can define a denotation of the \lst{SigmaProp} type (i.e. a set of possible values)

$$\Denot{\lst{SigmaProp}} = \Set{t \in Tree}$$


The following methods can be called on all instances of \lst{SigmaProp} type.


\subsubsection{\lst{SigmaProp.propBytes} method (Code 8.1)}
\label{sec:type:SigmaProp:propBytes}
\noindent
\begin{tabularx}{\textwidth}{| l | X |}
   \hline
   \bf{Description} & Serialized bytes of this sigma proposition taken as ErgoTree. \\
  
  \hline
  \bf{Parameters} &
      \(\begin{array}{l l l}
         
      \end{array}\) \\
       
  \hline
  \bf{Result} & \lst{Coll[Byte]} \\
  \hline
  
  \bf{Serialized as} & \hyperref[sec:serialization:operation:SigmaPropBytes]{\lst{SigmaPropBytes}} \\
  \hline
       
\end{tabularx}



\subsubsection{\lst{SigmaProp.isProven} method (Code 8.2)}
\label{sec:type:SigmaProp:isProven}
\noindent
\begin{tabularx}{\textwidth}{| l | X |}
   \hline
   \bf{Description} & Verify that sigma proposition is proven. (FRONTEND ONLY) \\
  
  \hline
  \bf{Parameters} &
      \(\begin{array}{l l l}
         
      \end{array}\) \\
       
  \hline
  \bf{Result} & \lst{Boolean} \\
  \hline
  
\end{tabularx}


Additionally, for a list of primitive operations on \lst{SigmaProp} type see
Appendix~\ref{sec:appendix:primops}.

\subsection{Box type}
\label{sec:type:Box}
 
\subsubsection{\lst{Box.value} method (Code 99.1)}
\label{sec:type:Box:value}
\noindent
\begin{tabularx}{\textwidth}{| l | X |}
   \hline
   \bf{Description} & Monetary value in NanoERGs stored in this box. \\
   \hline
   \bf{Signature} & \lst{def value}: \lst{Long} \\
  
  \hline
  
  \bf{Serialized as} & \hyperref[sec:serialization:operation:ExtractAmount]{\lst{ExtractAmount}} \\
  \hline
       
\end{tabularx}



\subsubsection{\lst{Box.propositionBytes} method (Code 99.2)}
\label{sec:type:Box:propositionBytes}
\noindent
\begin{tabularx}{\textwidth}{| l | X |}
   \hline
   \bf{Description} & Serialized bytes of the guarding script which should be evaluated to true in order to
 open this box (spend it in a transaction). \\
   \hline
   \bf{Signature} & \lst{def propositionBytes}: \lst{Coll[Byte]} \\
  
  \hline
  
  \bf{Serialized as} & \hyperref[sec:serialization:operation:ExtractScriptBytes]{\lst{ExtractScriptBytes}} \\
  \hline
       
\end{tabularx}



\subsubsection{\lst{Box.bytes} method (Code 99.3)}
\label{sec:type:Box:bytes}
\noindent
\begin{tabularx}{\textwidth}{| l | X |}
   \hline
   \bf{Description} & Serialized bytes of this box's content, including proposition bytes. \\
   \hline
   \bf{Signature} & \lst{def bytes}: \lst{Coll[Byte]} \\
  
  \hline
  
  \bf{Serialized as} & \hyperref[sec:serialization:operation:ExtractBytes]{\lst{ExtractBytes}} \\
  \hline
       
\end{tabularx}



\subsubsection{\lst{Box.bytesWithoutRef} method (Code 99.4)}
\label{sec:type:Box:bytesWithoutRef}
\noindent
\begin{tabularx}{\textwidth}{| l | X |}
   \hline
   \bf{Description} & Serialized bytes of this box's content, excluding transactionId and index of output. \\
   \hline
   \bf{Signature} & \lst{def bytesWithoutRef}: \lst{Coll[Byte]} \\
  
  \hline
  
  \bf{Serialized as} & \hyperref[sec:serialization:operation:ExtractBytesWithNoRef]{\lst{ExtractBytesWithNoRef}} \\
  \hline
       
\end{tabularx}



\subsubsection{\lst{Box.id} method (Code 99.5)}
\label{sec:type:Box:id}
\noindent
\begin{tabularx}{\textwidth}{| l | X |}
   \hline
   \bf{Description} & Blake2b256 hash of this box's content, basically equals to \lst{blake2b256(bytes)} \\
   \hline
   \bf{Signature} & \lst{def id}: \lst{Coll[Byte]} \\
  
  \hline
  
  \bf{Serialized as} & \hyperref[sec:serialization:operation:ExtractId]{\lst{ExtractId}} \\
  \hline
       
\end{tabularx}



\subsubsection{\lst{Box.creationInfo} method (Code 99.6)}
\label{sec:type:Box:creationInfo}
\noindent
\begin{tabularx}{\textwidth}{| l | X |}
   \hline
   \bf{Description} &  If \lst{tx} is a transaction which generated this box, then \lst{creationInfo._1}
 is a height of the tx's block. The \lst{creationInfo._2} is a serialized bytes of the transaction
 identifier followed by the serialized bytes of the box index in the transaction outputs.
         \\
   \hline
   \bf{Signature} & \lst{def creationInfo}: \lst{(Int,Coll[Byte])} \\
  
  \hline
  
  \bf{Serialized as} & \hyperref[sec:serialization:operation:ExtractCreationInfo]{\lst{ExtractCreationInfo}} \\
  \hline
       
\end{tabularx}



\subsubsection{\lst{Box.tokens} method (Code 99.8)}
\label{sec:type:Box:tokens}
\noindent
\begin{tabularx}{\textwidth}{| l | X |}
   \hline
   \bf{Description} & Secondary tokens \\
   \hline
   \bf{Signature} & \lst{def tokens}: \lst{Coll[(Coll[Byte],Long)]} \\
  
  \hline
  
  \bf{Serialized as} & \hyperref[sec:serialization:operation:PropertyCall]{\lst{PropertyCall}} \\
  \hline
       
\end{tabularx}



\subsubsection{\lst{Box.R4} method (Code 99.13)}
\label{sec:type:Box:R4}
\noindent
\begin{tabularx}{\textwidth}{| l | X |}
   \hline
   \bf{Description} & Non-mandatory register \\
   \hline
   \bf{Signature} & \lst{def R4}$[$\lst{T}$]$: \lst{Option[T]} \\
  
  \hline
  
  \bf{Serialized as} & \hyperref[sec:serialization:operation:ExtractRegisterAs]{\lst{ExtractRegisterAs}} \\
  \hline
       
\end{tabularx}



\subsubsection{\lst{Box.R5} method (Code 99.14)}
\label{sec:type:Box:R5}
\noindent
\begin{tabularx}{\textwidth}{| l | X |}
   \hline
   \bf{Description} & Non-mandatory register \\
   \hline
   \bf{Signature} & \lst{def R5}$[$\lst{T}$]$: \lst{Option[T]} \\
  
  \hline
  
  \bf{Serialized as} & \hyperref[sec:serialization:operation:ExtractRegisterAs]{\lst{ExtractRegisterAs}} \\
  \hline
       
\end{tabularx}



\subsubsection{\lst{Box.R6} method (Code 99.15)}
\label{sec:type:Box:R6}
\noindent
\begin{tabularx}{\textwidth}{| l | X |}
   \hline
   \bf{Description} & Non-mandatory register \\
   \hline
   \bf{Signature} & \lst{def R6}$[$\lst{T}$]$: \lst{Option[T]} \\
  
  \hline
  
  \bf{Serialized as} & \hyperref[sec:serialization:operation:ExtractRegisterAs]{\lst{ExtractRegisterAs}} \\
  \hline
       
\end{tabularx}



\subsubsection{\lst{Box.R7} method (Code 99.16)}
\label{sec:type:Box:R7}
\noindent
\begin{tabularx}{\textwidth}{| l | X |}
   \hline
   \bf{Description} & Non-mandatory register \\
   \hline
   \bf{Signature} & \lst{def R7}$[$\lst{T}$]$: \lst{Option[T]} \\
  
  \hline
  
  \bf{Serialized as} & \hyperref[sec:serialization:operation:ExtractRegisterAs]{\lst{ExtractRegisterAs}} \\
  \hline
       
\end{tabularx}



\subsubsection{\lst{Box.R8} method (Code 99.17)}
\label{sec:type:Box:R8}
\noindent
\begin{tabularx}{\textwidth}{| l | X |}
   \hline
   \bf{Description} & Non-mandatory register \\
   \hline
   \bf{Signature} & \lst{def R8}$[$\lst{T}$]$: \lst{Option[T]} \\
  
  \hline
  
  \bf{Serialized as} & \hyperref[sec:serialization:operation:ExtractRegisterAs]{\lst{ExtractRegisterAs}} \\
  \hline
       
\end{tabularx}



\subsubsection{\lst{Box.R9} method (Code 99.18)}
\label{sec:type:Box:R9}
\noindent
\begin{tabularx}{\textwidth}{| l | X |}
   \hline
   \bf{Description} & Non-mandatory register \\
   \hline
   \bf{Signature} & \lst{def R9}$[$\lst{T}$]$: \lst{Option[T]} \\
  
  \hline
  
  \bf{Serialized as} & \hyperref[sec:serialization:operation:ExtractRegisterAs]{\lst{ExtractRegisterAs}} \\
  \hline
       
\end{tabularx}


\subsection{\lst{AvlTree} type}
\label{sec:type:AvlTree}
 
\subsubsection{\lst{AvlTree.digest} method (Code 100.1)}
\label{sec:type:AvlTree:digest}
\noindent
\begin{tabularx}{\textwidth}{| l | X |}
   \hline
   \bf{Description} & Returns digest of the state represented by this tree.
 Authenticated tree \lst{digest} = \lst{root hash bytes} ++ \lst{tree height}
         \\
  
  \hline
  \bf{Parameters} &
      \(\begin{array}{l l l}
         
      \end{array}\) \\
       
  \hline
  \bf{Result} & \lst{Coll[Byte]} \\
  \hline
  
  \bf{Serialized as} & \hyperref[sec:serialization:operation:PropertyCall]{\lst{PropertyCall}} \\
  \hline
       
\end{tabularx}



\subsubsection{\lst{AvlTree.enabledOperations} method (Code 100.2)}
\label{sec:type:AvlTree:enabledOperations}
\noindent
\begin{tabularx}{\textwidth}{| l | X |}
   \hline
   \bf{Description} &  Flags of enabled operations packed in single byte.
 \lst{isInsertAllowed == (enabledOperations & 0x01) != 0}\newline
 \lst{isUpdateAllowed == (enabledOperations & 0x02) != 0}\newline
 \lst{isRemoveAllowed == (enabledOperations & 0x04) != 0}
         \\
  
  \hline
  \bf{Parameters} &
      \(\begin{array}{l l l}
         
      \end{array}\) \\
       
  \hline
  \bf{Result} & \lst{Byte} \\
  \hline
  
  \bf{Serialized as} & \hyperref[sec:serialization:operation:PropertyCall]{\lst{PropertyCall}} \\
  \hline
       
\end{tabularx}



\subsubsection{\lst{AvlTree.keyLength} method (Code 100.3)}
\label{sec:type:AvlTree:keyLength}
\noindent
\begin{tabularx}{\textwidth}{| l | X |}
   \hline
   \bf{Description} & 

             \\
  
  \hline
  \bf{Parameters} &
      \(\begin{array}{l l l}
         
      \end{array}\) \\
       
  \hline
  \bf{Result} & \lst{Int} \\
  \hline
  
  \bf{Serialized as} & \hyperref[sec:serialization:operation:PropertyCall]{\lst{PropertyCall}} \\
  \hline
       
\end{tabularx}



\subsubsection{\lst{AvlTree.valueLengthOpt} method (Code 100.4)}
\label{sec:type:AvlTree:valueLengthOpt}
\noindent
\begin{tabularx}{\textwidth}{| l | X |}
   \hline
   \bf{Description} & 

         \\
  
  \hline
  \bf{Parameters} &
      \(\begin{array}{l l l}
         
      \end{array}\) \\
       
  \hline
  \bf{Result} & \lst{Option[Int]} \\
  \hline
  
  \bf{Serialized as} & \hyperref[sec:serialization:operation:PropertyCall]{\lst{PropertyCall}} \\
  \hline
       
\end{tabularx}



\subsubsection{\lst{AvlTree.isInsertAllowed} method (Code 100.5)}
\label{sec:type:AvlTree:isInsertAllowed}
\noindent
\begin{tabularx}{\textwidth}{| l | X |}
   \hline
   \bf{Description} & 

         \\
  
  \hline
  \bf{Parameters} &
      \(\begin{array}{l l l}
         
      \end{array}\) \\
       
  \hline
  \bf{Result} & \lst{Boolean} \\
  \hline
  
  \bf{Serialized as} & \hyperref[sec:serialization:operation:PropertyCall]{\lst{PropertyCall}} \\
  \hline
       
\end{tabularx}



\subsubsection{\lst{AvlTree.isUpdateAllowed} method (Code 100.6)}
\label{sec:type:AvlTree:isUpdateAllowed}
\noindent
\begin{tabularx}{\textwidth}{| l | X |}
   \hline
   \bf{Description} & 

         \\
  
  \hline
  \bf{Parameters} &
      \(\begin{array}{l l l}
         
      \end{array}\) \\
       
  \hline
  \bf{Result} & \lst{Boolean} \\
  \hline
  
  \bf{Serialized as} & \hyperref[sec:serialization:operation:PropertyCall]{\lst{PropertyCall}} \\
  \hline
       
\end{tabularx}



\subsubsection{\lst{AvlTree.isRemoveAllowed} method (Code 100.7)}
\label{sec:type:AvlTree:isRemoveAllowed}
\noindent
\begin{tabularx}{\textwidth}{| l | X |}
   \hline
   \bf{Description} & 

         \\
  
  \hline
  \bf{Parameters} &
      \(\begin{array}{l l l}
         
      \end{array}\) \\
       
  \hline
  \bf{Result} & \lst{Boolean} \\
  \hline
  
  \bf{Serialized as} & \hyperref[sec:serialization:operation:PropertyCall]{\lst{PropertyCall}} \\
  \hline
       
\end{tabularx}



\subsubsection{\lst{AvlTree.updateOperations} method (Code 100.8)}
\label{sec:type:AvlTree:updateOperations}
\noindent
\begin{tabularx}{\textwidth}{| l | X |}
   \hline
   \bf{Description} & 

         \\
  
  \hline
  \bf{Parameters} &
      \(\begin{array}{l l l}
         
      \end{array}\) \\
       
  \hline
  \bf{Result} & \lst{AvlTree} \\
  \hline
  
  \bf{Serialized as} & \hyperref[sec:serialization:operation:MethodCall]{\lst{MethodCall}} \\
  \hline
       
\end{tabularx}



\subsubsection{\lst{AvlTree.contains} method (Code 100.9)}
\label{sec:type:AvlTree:contains}
\noindent
\begin{tabularx}{\textwidth}{| l | X |}
   \hline
   \bf{Description} & 
   /** Checks if an entry with key `key` exists in this tree using proof `proof`.
    * Throws exception if proof is incorrect

    * @note CAUTION! Does not support multiple keys check, use [[getMany]] instead.
    * Return `true` if a leaf with the key `key` exists
    * Return `false` if leaf with provided key does not exist.
    * @param key    a key of an element of this authenticated dictionary.
    * @param proof
    */

         \\
  
  \hline
  \bf{Parameters} &
      \(\begin{array}{l l l}
         
      \end{array}\) \\
       
  \hline
  \bf{Result} & \lst{Boolean} \\
  \hline
  
  \bf{Serialized as} & \hyperref[sec:serialization:operation:MethodCall]{\lst{MethodCall}} \\
  \hline
       
\end{tabularx}



\subsubsection{\lst{AvlTree.get} method (Code 100.10)}
\label{sec:type:AvlTree:get}
\noindent
\begin{tabularx}{\textwidth}{| l | X |}
   \hline
   \bf{Description} & 
  /** Perform a lookup of key `key` in this tree using proof `proof`.
    * Throws exception if proof is incorrect
    *
    * @note CAUTION! Does not support multiple keys check, use [[getMany]] instead.
    * Return Some(bytes) of leaf with key `key` if it exists
    * Return None if leaf with provided key does not exist.
    * @param key    a key of an element of this authenticated dictionary.
    * @param proof
    */

         \\
  
  \hline
  \bf{Parameters} &
      \(\begin{array}{l l l}
         
      \end{array}\) \\
       
  \hline
  \bf{Result} & \lst{Option[Coll[Byte]]} \\
  \hline
  
  \bf{Serialized as} & \hyperref[sec:serialization:operation:MethodCall]{\lst{MethodCall}} \\
  \hline
       
\end{tabularx}



\subsubsection{\lst{AvlTree.getMany} method (Code 100.11)}
\label{sec:type:AvlTree:getMany}
\noindent
\begin{tabularx}{\textwidth}{| l | X |}
   \hline
   \bf{Description} & 
  /** Perform a lookup of many keys `keys` in this tree using proof `proof`.
    *
    * @note CAUTION! Keys must be ordered the same way they were in lookup before proof was generated.
    * For each key return Some(bytes) of leaf if it exists and None if is doesn't.
    * @param keys    keys of elements of this authenticated dictionary.
    * @param proof
    */

         \\
  
  \hline
  \bf{Parameters} &
      \(\begin{array}{l l l}
         
      \end{array}\) \\
       
  \hline
  \bf{Result} & \lst{Coll[Option[Coll[Byte]]]} \\
  \hline
  
  \bf{Serialized as} & \hyperref[sec:serialization:operation:MethodCall]{\lst{MethodCall}} \\
  \hline
       
\end{tabularx}



\subsubsection{\lst{AvlTree.insert} method (Code 100.12)}
\label{sec:type:AvlTree:insert}
\noindent
\begin{tabularx}{\textwidth}{| l | X |}
   \hline
   \bf{Description} & 
  /** Perform insertions of key-value entries into this tree using proof `proof`.
    * Throws exception if proof is incorrect
    *
    * @note CAUTION! Pairs must be ordered the same way they were in insert ops before proof was generated.
    * Return Some(newTree) if successful
    * Return None if operations were not performed.
    * @param operations   collection of key-value pairs to insert in this authenticated dictionary.
    * @param proof
    */

         \\
  
  \hline
  \bf{Parameters} &
      \(\begin{array}{l l l}
         
      \end{array}\) \\
       
  \hline
  \bf{Result} & \lst{Option[AvlTree]} \\
  \hline
  
  \bf{Serialized as} & \hyperref[sec:serialization:operation:MethodCall]{\lst{MethodCall}} \\
  \hline
       
\end{tabularx}



\subsubsection{\lst{AvlTree.update} method (Code 100.13)}
\label{sec:type:AvlTree:update}
\noindent
\begin{tabularx}{\textwidth}{| l | X |}
   \hline
   \bf{Description} & 
  /** Perform updates of key-value entries into this tree using proof `proof`.
    * Throws exception if proof is incorrect
    *
    * @note CAUTION! Pairs must be ordered the same way they were in update ops before proof was generated.
    * Return Some(newTree) if successful
    * Return None if operations were not performed.
    * @param operations   collection of key-value pairs to update in this authenticated dictionary.
    * @param proof
    */

         \\
  
  \hline
  \bf{Parameters} &
      \(\begin{array}{l l l}
         
      \end{array}\) \\
       
  \hline
  \bf{Result} & \lst{Option[AvlTree]} \\
  \hline
  
  \bf{Serialized as} & \hyperref[sec:serialization:operation:MethodCall]{\lst{MethodCall}} \\
  \hline
       
\end{tabularx}



\subsubsection{\lst{AvlTree.remove} method (Code 100.14)}
\label{sec:type:AvlTree:remove}
\noindent
\begin{tabularx}{\textwidth}{| l | X |}
   \hline
   \bf{Description} & 
  /** Perform removal of entries into this tree using proof `proof`.
    * Throws exception if proof is incorrect
    * Return Some(newTree) if successful
    * Return None if operations were not performed.
    *
    * @note CAUTION! Keys must be ordered the same way they were in remove ops before proof was generated.
    * @param operations   collection of keys to remove from this authenticated dictionary.
    * @param proof
    */

         \\
  
  \hline
  \bf{Parameters} &
      \(\begin{array}{l l l}
         
      \end{array}\) \\
       
  \hline
  \bf{Result} & \lst{Option[AvlTree]} \\
  \hline
  
  \bf{Serialized as} & \hyperref[sec:serialization:operation:MethodCall]{\lst{MethodCall}} \\
  \hline
       
\end{tabularx}



\subsubsection{\lst{AvlTree.updateDigest} method (Code 100.15)}
\label{sec:type:AvlTree:updateDigest}
\noindent
\begin{tabularx}{\textwidth}{| l | X |}
   \hline
   \bf{Description} & 

         \\
  
  \hline
  \bf{Parameters} &
      \(\begin{array}{l l l}
         
      \end{array}\) \\
       
  \hline
  \bf{Result} & \lst{AvlTree} \\
  \hline
  
  \bf{Serialized as} & \hyperref[sec:serialization:operation:MethodCall]{\lst{MethodCall}} \\
  \hline
       
\end{tabularx}


\subsection{Header type}
\label{sec:type:Header}
 
\subsubsection{\lst{Header.id} method (Code 104.1)}
\label{sec:type:Header:id}
\noindent
\begin{tabularx}{\textwidth}{| l | X |}
   \hline
   \bf{Description} & Bytes representation of ModifierId of this Header \\
   \hline
   \bf{Signature} & \lst{def id}: \lst{Coll[Byte]} \\
  
  \hline
  
  \bf{Serialized as} & \hyperref[sec:serialization:operation:PropertyCall]{\lst{PropertyCall}} \\
  \hline
       
\end{tabularx}



\subsubsection{\lst{Header.version} method (Code 104.2)}
\label{sec:type:Header:version}
\noindent
\begin{tabularx}{\textwidth}{| l | X |}
   \hline
   \bf{Description} & Block version, to be increased on every soft and hard-fork. \\
   \hline
   \bf{Signature} & \lst{def version}: \lst{Byte} \\
  
  \hline
  
  \bf{Serialized as} & \hyperref[sec:serialization:operation:PropertyCall]{\lst{PropertyCall}} \\
  \hline
       
\end{tabularx}



\subsubsection{\lst{Header.parentId} method (Code 104.3)}
\label{sec:type:Header:parentId}
\noindent
\begin{tabularx}{\textwidth}{| l | X |}
   \hline
   \bf{Description} & Bytes representation of ModifierId of the parent block \\
   \hline
   \bf{Signature} & \lst{def parentId}: \lst{Coll[Byte]} \\
  
  \hline
  
  \bf{Serialized as} & \hyperref[sec:serialization:operation:PropertyCall]{\lst{PropertyCall}} \\
  \hline
       
\end{tabularx}



\subsubsection{\lst{Header.ADProofsRoot} method (Code 104.4)}
\label{sec:type:Header:ADProofsRoot}
\noindent
\begin{tabularx}{\textwidth}{| l | X |}
   \hline
   \bf{Description} & Hash of ADProofs for transactions in a block \\
   \hline
   \bf{Signature} & \lst{def ADProofsRoot}: \lst{Coll[Byte]} \\
  
  \hline
  
  \bf{Serialized as} & \hyperref[sec:serialization:operation:PropertyCall]{\lst{PropertyCall}} \\
  \hline
       
\end{tabularx}



\subsubsection{\lst{Header.stateRoot} method (Code 104.5)}
\label{sec:type:Header:stateRoot}
\noindent
\begin{tabularx}{\textwidth}{| l | X |}
   \hline
   \bf{Description} & AvlTree of a state after block application \\
   \hline
   \bf{Signature} & \lst{def stateRoot}: \lst{AvlTree} \\
  
  \hline
  
  \bf{Serialized as} & \hyperref[sec:serialization:operation:PropertyCall]{\lst{PropertyCall}} \\
  \hline
       
\end{tabularx}



\subsubsection{\lst{Header.transactionsRoot} method (Code 104.6)}
\label{sec:type:Header:transactionsRoot}
\noindent
\begin{tabularx}{\textwidth}{| l | X |}
   \hline
   \bf{Description} & Root hash (for a Merkle tree) of transactions in a block. \\
   \hline
   \bf{Signature} & \lst{def transactionsRoot}: \lst{Coll[Byte]} \\
  
  \hline
  
  \bf{Serialized as} & \hyperref[sec:serialization:operation:PropertyCall]{\lst{PropertyCall}} \\
  \hline
       
\end{tabularx}



\subsubsection{\lst{Header.timestamp} method (Code 104.7)}
\label{sec:type:Header:timestamp}
\noindent
\begin{tabularx}{\textwidth}{| l | X |}
   \hline
   \bf{Description} & Block timestamp (in milliseconds since beginning of Unix Epoch) \\
   \hline
   \bf{Signature} & \lst{def timestamp}: \lst{Long} \\
  
  \hline
  
  \bf{Serialized as} & \hyperref[sec:serialization:operation:PropertyCall]{\lst{PropertyCall}} \\
  \hline
       
\end{tabularx}



\subsubsection{\lst{Header.nBits} method (Code 104.8)}
\label{sec:type:Header:nBits}
\noindent
\begin{tabularx}{\textwidth}{| l | X |}
   \hline
   \bf{Description} & Current difficulty in a compressed view. NOTE: actually it is unsigned Int. \\
   \hline
   \bf{Signature} & \lst{def nBits}: \lst{Long} \\
  
  \hline
  
  \bf{Serialized as} & \hyperref[sec:serialization:operation:PropertyCall]{\lst{PropertyCall}} \\
  \hline
       
\end{tabularx}



\subsubsection{\lst{Header.height} method (Code 104.9)}
\label{sec:type:Header:height}
\noindent
\begin{tabularx}{\textwidth}{| l | X |}
   \hline
   \bf{Description} & Block height \\
   \hline
   \bf{Signature} & \lst{def height}: \lst{Int} \\
  
  \hline
  
  \bf{Serialized as} & \hyperref[sec:serialization:operation:PropertyCall]{\lst{PropertyCall}} \\
  \hline
       
\end{tabularx}



\subsubsection{\lst{Header.extensionRoot} method (Code 104.10)}
\label{sec:type:Header:extensionRoot}
\noindent
\begin{tabularx}{\textwidth}{| l | X |}
   \hline
   \bf{Description} & Root hash of extension section \\
   \hline
   \bf{Signature} & \lst{def extensionRoot}: \lst{Coll[Byte]} \\
  
  \hline
  
  \bf{Serialized as} & \hyperref[sec:serialization:operation:PropertyCall]{\lst{PropertyCall}} \\
  \hline
       
\end{tabularx}



\subsubsection{\lst{Header.minerPk} method (Code 104.11)}
\label{sec:type:Header:minerPk}
\noindent
\begin{tabularx}{\textwidth}{| l | X |}
   \hline
   \bf{Description} & Miner public key. Should be used to collect block rewards. Part of Autolykos solution. \\
   \hline
   \bf{Signature} & \lst{def minerPk}: \lst{GroupElement} \\
  
  \hline
  
  \bf{Serialized as} & \hyperref[sec:serialization:operation:PropertyCall]{\lst{PropertyCall}} \\
  \hline
       
\end{tabularx}



\subsubsection{\lst{Header.powOnetimePk} method (Code 104.12)}
\label{sec:type:Header:powOnetimePk}
\noindent
\begin{tabularx}{\textwidth}{| l | X |}
   \hline
   \bf{Description} & One-time public key. Prevents revealing of miners secret. \\
   \hline
   \bf{Signature} & \lst{def powOnetimePk}: \lst{GroupElement} \\
  
  \hline
  
  \bf{Serialized as} & \hyperref[sec:serialization:operation:PropertyCall]{\lst{PropertyCall}} \\
  \hline
       
\end{tabularx}



\subsubsection{\lst{Header.powNonce} method (Code 104.13)}
\label{sec:type:Header:powNonce}
\noindent
\begin{tabularx}{\textwidth}{| l | X |}
   \hline
   \bf{Description} & The nonce value generated during mining. \\
   \hline
   \bf{Signature} & \lst{def powNonce}: \lst{Coll[Byte]} \\
  
  \hline
  
  \bf{Serialized as} & \hyperref[sec:serialization:operation:PropertyCall]{\lst{PropertyCall}} \\
  \hline
       
\end{tabularx}



\subsubsection{\lst{Header.powDistance} method (Code 104.14)}
\label{sec:type:Header:powDistance}
\noindent
\begin{tabularx}{\textwidth}{| l | X |}
   \hline
   \bf{Description} & Distance between pseudo-random number, corresponding to nonce \lst{powNonce} and a secret,
corresponding to \lst{minerPk}. The lower \lst{powDistance} is, the harder it was to find this solution. \\
   \hline
   \bf{Signature} & \lst{def powDistance}: \lst{BigInt} \\
  
  \hline
  
  \bf{Serialized as} & \hyperref[sec:serialization:operation:PropertyCall]{\lst{PropertyCall}} \\
  \hline
       
\end{tabularx}



\subsubsection{\lst{Header.votes} method (Code 104.15)}
\label{sec:type:Header:votes}
\noindent
\begin{tabularx}{\textwidth}{| l | X |}
   \hline
   \bf{Description} & A collection of votes set up by the block miner. \\
   \hline
   \bf{Signature} & \lst{def votes}: \lst{Coll[Byte]} \\
  
  \hline
  
  \bf{Serialized as} & \hyperref[sec:serialization:operation:PropertyCall]{\lst{PropertyCall}} \\
  \hline
       
\end{tabularx}


\subsection{PreHeader type}
\label{sec:type:PreHeader}
 
\subsubsection{\lst{PreHeader.version} method (Code 105.1)}
\label{sec:type:PreHeader:version}
\noindent
\begin{tabularx}{\textwidth}{| l | X |}
   \hline
   \bf{Description} & Block version, to be increased on every soft and hard-fork. \\
   \hline
   \bf{Signature} & \lst{def version}: \lst{Byte} \\
  
  \hline
  
  \bf{Serialized as} & \hyperref[sec:serialization:operation:PropertyCall]{\lst{PropertyCall}} \\
  \hline
       
\end{tabularx}



\subsubsection{\lst{PreHeader.parentId} method (Code 105.2)}
\label{sec:type:PreHeader:parentId}
\noindent
\begin{tabularx}{\textwidth}{| l | X |}
   \hline
   \bf{Description} & Id of parent block \\
   \hline
   \bf{Signature} & \lst{def parentId}: \lst{Coll[Byte]} \\
  
  \hline
  
  \bf{Serialized as} & \hyperref[sec:serialization:operation:PropertyCall]{\lst{PropertyCall}} \\
  \hline
       
\end{tabularx}



\subsubsection{\lst{PreHeader.timestamp} method (Code 105.3)}
\label{sec:type:PreHeader:timestamp}
\noindent
\begin{tabularx}{\textwidth}{| l | X |}
   \hline
   \bf{Description} & Block timestamp (in milliseconds since beginning of Unix Epoch) \\
   \hline
   \bf{Signature} & \lst{def timestamp}: \lst{Long} \\
  
  \hline
  
  \bf{Serialized as} & \hyperref[sec:serialization:operation:PropertyCall]{\lst{PropertyCall}} \\
  \hline
       
\end{tabularx}



\subsubsection{\lst{PreHeader.nBits} method (Code 105.4)}
\label{sec:type:PreHeader:nBits}
\noindent
\begin{tabularx}{\textwidth}{| l | X |}
   \hline
   \bf{Description} & Current difficulty in a compressed view. NOTE: actually it is unsigned Int. \\
   \hline
   \bf{Signature} & \lst{def nBits}: \lst{Long} \\
  
  \hline
  
  \bf{Serialized as} & \hyperref[sec:serialization:operation:PropertyCall]{\lst{PropertyCall}} \\
  \hline
       
\end{tabularx}



\subsubsection{\lst{PreHeader.height} method (Code 105.5)}
\label{sec:type:PreHeader:height}
\noindent
\begin{tabularx}{\textwidth}{| l | X |}
   \hline
   \bf{Description} & Block height \\
   \hline
   \bf{Signature} & \lst{def height}: \lst{Int} \\
  
  \hline
  
  \bf{Serialized as} & \hyperref[sec:serialization:operation:PropertyCall]{\lst{PropertyCall}} \\
  \hline
       
\end{tabularx}



\subsubsection{\lst{PreHeader.minerPk} method (Code 105.6)}
\label{sec:type:PreHeader:minerPk}
\noindent
\begin{tabularx}{\textwidth}{| l | X |}
   \hline
   \bf{Description} & Miner public key. Should be used to collect block rewards. \\
   \hline
   \bf{Signature} & \lst{def minerPk}: \lst{GroupElement} \\
  
  \hline
  
  \bf{Serialized as} & \hyperref[sec:serialization:operation:PropertyCall]{\lst{PropertyCall}} \\
  \hline
       
\end{tabularx}



\subsubsection{\lst{PreHeader.votes} method (Code 105.7)}
\label{sec:type:PreHeader:votes}
\noindent
\begin{tabularx}{\textwidth}{| l | X |}
   \hline
   \bf{Description} & A collection of votes set up by the block miner. \\
   \hline
   \bf{Signature} & \lst{def votes}: \lst{Coll[Byte]} \\
  
  \hline
  
  \bf{Serialized as} & \hyperref[sec:serialization:operation:PropertyCall]{\lst{PropertyCall}} \\
  \hline
       
\end{tabularx}


\subsection{Context type}
\label{sec:type:Context}

\subsubsection{\lst{Context.dataInputs} method (Code 101.1)}
\label{sec:type:Context:dataInputs}
\noindent
\begin{tabularx}{\textwidth}{| l | X |}
   \hline
   \bf{Description} & A collection of inputs of the current transaction that will not be spent. \\
   \hline
   \bf{Signature} & \lst{def dataInputs}: \lst{Coll[Box]} \\
  
  \hline
  
  \bf{Serialized as} & \hyperref[sec:serialization:operation:PropertyCall]{\lst{PropertyCall}} \\
  \hline
       
\end{tabularx}



\subsubsection{\lst{Context.headers} method (Code 101.2)}
\label{sec:type:Context:headers}
\noindent
\begin{tabularx}{\textwidth}{| l | X |}
   \hline
   \bf{Description} & A fixed number of last block headers in descending order (first header is the newest one) \\
   \hline
   \bf{Signature} & \lst{def headers}: \lst{Coll[Header]} \\
  
  \hline
  
  \bf{Serialized as} & \hyperref[sec:serialization:operation:PropertyCall]{\lst{PropertyCall}} \\
  \hline
       
\end{tabularx}



\subsubsection{\lst{Context.preHeader} method (Code 101.3)}
\label{sec:type:Context:preHeader}
\noindent
\begin{tabularx}{\textwidth}{| l | X |}
   \hline
   \bf{Description} & Only header fields that can be predicted by a miner when the spending transaction is added to a new block candidate. \\
   \hline
   \bf{Signature} & \lst{def preHeader}: \lst{PreHeader} \\
  
  \hline
  
  \bf{Serialized as} & \hyperref[sec:serialization:operation:PropertyCall]{\lst{PropertyCall}} \\
  \hline
       
\end{tabularx}



\subsubsection{\lst{Context.INPUTS} method (Code 101.4)}
\label{sec:type:Context:INPUTS}
\noindent
\begin{tabularx}{\textwidth}{| l | X |}
   \hline
   \bf{Description} & A collection of inputs of the current transaction,
where the \lst{SELF} box is one of the inputs. \\
   \hline
   \bf{Signature} & \lst{def INPUTS}: \lst{Coll[Box]} \\
  
  \hline
  
  \bf{Serialized as} & \hyperref[sec:serialization:operation:Inputs]{\lst{Inputs}} \\
  \hline
       
\end{tabularx}



\subsubsection{\lst{Context.OUTPUTS} method (Code 101.5)}
\label{sec:type:Context:OUTPUTS}
\noindent
\begin{tabularx}{\textwidth}{| l | X |}
   \hline
   \bf{Description} & A collection of outputs of the current transaction. \\
   \hline
   \bf{Signature} & \lst{def OUTPUTS}: \lst{Coll[Box]} \\
  
  \hline
  
  \bf{Serialized as} & \hyperref[sec:serialization:operation:Outputs]{\lst{Outputs}} \\
  \hline
       
\end{tabularx}



\subsubsection{\lst{Context.HEIGHT} method (Code 101.6)}
\label{sec:type:Context:HEIGHT}
\noindent
\begin{tabularx}{\textwidth}{| l | X |}
   \hline
   \bf{Description} & Height (block number) of the block which is currently being validated. \\
   \hline
   \bf{Signature} & \lst{def HEIGHT}: \lst{Int} \\
  
  \hline
  
  \bf{Serialized as} & \hyperref[sec:serialization:operation:Height]{\lst{Height}} \\
  \hline
       
\end{tabularx}



\subsubsection{\lst{Context.SELF} method (Code 101.7)}
\label{sec:type:Context:SELF}
\noindent
\begin{tabularx}{\textwidth}{| l | X |}
   \hline
   \bf{Description} & Box whose proposition is being currently executing \\
   \hline
   \bf{Signature} & \lst{def SELF}: \lst{Box} \\
  
  \hline
  
  \bf{Serialized as} & \hyperref[sec:serialization:operation:Self]{\lst{Self}} \\
  \hline
       
\end{tabularx}



\subsubsection{\lst{Context.LastBlockUtxoRootHash} method (Code 101.9)}
\label{sec:type:Context:LastBlockUtxoRootHash}
\noindent
\begin{tabularx}{\textwidth}{| l | X |}
   \hline
   \bf{Description} & Authenticated dynamic dictionary digest representing Utxo state before current state. \\
   \hline
   \bf{Signature} & \lst{def LastBlockUtxoRootHash}: \lst{AvlTree} \\
  
  \hline
  
  \bf{Serialized as} & \hyperref[sec:serialization:operation:LastBlockUtxoRootHash]{\lst{LastBlockUtxoRootHash}} \\
  \hline
       
\end{tabularx}



\subsubsection{\lst{Context.minerPubKey} method (Code 101.10)}
\label{sec:type:Context:minerPubKey}
\noindent
\begin{tabularx}{\textwidth}{| l | X |}
   \hline
   \bf{Description} & Encoded bytes of public key of the miner who created the block.
Equals to \lst{preHeader.minerPk.getEncoded} \\
   \hline
   \bf{Signature} & \lst{def minerPubKey}: \lst{Coll[Byte]} \\
  
  \hline
  
  \bf{Serialized as} & \hyperref[sec:serialization:operation:MinerPubkey]{\lst{MinerPubkey}} \\
  \hline
       
\end{tabularx}



\subsubsection{\lst{Context.getVar} method (Code 101.11)}
\label{sec:type:Context:getVar}
\noindent
\begin{tabularx}{\textwidth}{| l | X |}
   \hline
   \bf{Description} & Get context variable with given \lst{varId} and type.
Example: \lst{getVar[Coll[Byte]](10).get} extract a collection of bytes
from the variable with varId = 10. \\
   \hline
   \bf{Signature} & \lst{def getVar}$[$\lst{T}$]$(\lst{varId}$:$~\lst{Byte}): \lst{Option[T]} \\
  
  \hline
  \bf{Parameters} &
      \(\begin{array}{l l}
         \lst{varId} & \text{\lst{Byte} identifier of context variable} \\
      \end{array}\) \\
       
  \hline
  
  \bf{Serialized as} & \hyperref[sec:serialization:operation:GetVar]{\lst{GetVar}} \\
  \hline
       
\end{tabularx}


\subsection{Global type}
\label{sec:type:Global}

\subsubsection{\lst{SigmaDslBuilder.groupGenerator} method (Code 106.1)}
\label{sec:type:SigmaDslBuilder:groupGenerator}
\noindent
\begin{tabularx}{\textwidth}{| l | X |}
   \hline
   \bf{Description} & The generator $g$ of the group is an element of the group such that,
when written multiplicatively, every element of the group is a power of $g$.
Returns the generator of the SecP256K1 group. \\
   \hline
   \bf{Signature} & \lst{def groupGenerator}: \lst{GroupElement} \\
  
  \hline
  
  \bf{Serialized as} & \hyperref[sec:serialization:operation:GroupGenerator]{\lst{GroupGenerator}} \\
  \hline
       
\end{tabularx}



\subsubsection{\lst{SigmaDslBuilder.xor} method (Code 106.2)}
\label{sec:type:SigmaDslBuilder:xor}
\noindent
\begin{tabularx}{\textwidth}{| l | X |}
   \hline
   \bf{Description} & Byte-wise XOR of two collections of bytes \\
   \hline
   \bf{Signature} & \lst{def xor}(\lst{left}$:$~\lst{Coll[Byte]}, \lst{right}$:$~\lst{Coll[Byte]}): \lst{Coll[Byte]} \\
  
  \hline
  \bf{Parameters} &
      \(\begin{array}{l l}
         \lst{left} & \text{left operand} \\
\lst{right} & \text{right operand} \\
      \end{array}\) \\
       
  \hline
  
  \bf{Serialized as} & \hyperref[sec:serialization:operation:Xor]{\lst{Xor}} \\
  \hline
       
\end{tabularx}


\subsection{Coll type}
\label{sec:type:Coll}
 
\subsubsection{\lst{SCollection.size} method (Code 12.1)}
\label{sec:type:SCollection:size}
\noindent
\begin{tabularx}{\textwidth}{| l | X |}
   \hline
   \bf{Description} & The size of the collection in elements. \\
   \hline
   \bf{Signature} & \lst{def size}: \lst{Int} \\
  
  \hline
  
  \bf{Serialized as} & \hyperref[sec:serialization:operation:SizeOf]{\lst{SizeOf}} \\
  \hline
       
\end{tabularx}



\subsubsection{\lst{SCollection.getOrElse} method (Code 12.2)}
\label{sec:type:SCollection:getOrElse}
\noindent
\begin{tabularx}{\textwidth}{| l | X |}
   \hline
   \bf{Description} & Return the element of collection if \lst{index} is in range \lst{0 .. size-1} \\
   \hline
   \bf{Signature} & \lst{def getOrElse}(\lst{index}$:$~\lst{Int}, \lst{default}$:$~\lst{IV}): \lst{IV} \\
  
  \hline
  \bf{Parameters} &
      \(\begin{array}{l l}
         \lst{index} & \text{index of the element of this collection} \\
\lst{default} & \text{value to return when \lst{index} is out of range} \\
      \end{array}\) \\
       
  \hline
  
  \bf{Serialized as} & \hyperref[sec:serialization:operation:ByIndex]{\lst{ByIndex}} \\
  \hline
       
\end{tabularx}



\subsubsection{\lst{SCollection.map} method (Code 12.3)}
\label{sec:type:SCollection:map}
\noindent
\begin{tabularx}{\textwidth}{| l | X |}
   \hline
   \bf{Description} &  Builds a new collection by applying a function to all elements of this collection.
 Returns a new collection of type \lst{Coll[B]} resulting from applying the given function
 \lst{f} to each element of this collection and collecting the results.
         \\
   \hline
   \bf{Signature} & \lst{def map}$[$\lst{OV}$]$(\lst{f}$:$~\lst{(IV) => OV}): \lst{Coll[OV]} \\
  
  \hline
  \bf{Parameters} &
      \(\begin{array}{l l}
         \lst{f} & \text{the function to apply to each element} \\
      \end{array}\) \\
       
  \hline
  
  \bf{Serialized as} & \hyperref[sec:serialization:operation:MapCollection]{\lst{MapCollection}} \\
  \hline
       
\end{tabularx}



\subsubsection{\lst{SCollection.exists} method (Code 12.4)}
\label{sec:type:SCollection:exists}
\noindent
\begin{tabularx}{\textwidth}{| l | X |}
   \hline
   \bf{Description} & Tests whether a predicate holds for at least one element of this collection.
Returns \lst{true} if the given predicate \lst{p} is satisfied by at least one element of this collection, otherwise \lst{false}
         \\
   \hline
   \bf{Signature} & \lst{def exists}(\lst{p}$:$~\lst{(IV) => Boolean}): \lst{Boolean} \\
  
  \hline
  \bf{Parameters} &
      \(\begin{array}{l l}
         \lst{p} & \text{the predicate used to test elements} \\
      \end{array}\) \\
       
  \hline
  
  \bf{Serialized as} & \hyperref[sec:serialization:operation:Exists]{\lst{Exists}} \\
  \hline
       
\end{tabularx}



\subsubsection{\lst{SCollection.fold} method (Code 12.5)}
\label{sec:type:SCollection:fold}
\noindent
\begin{tabularx}{\textwidth}{| l | X |}
   \hline
   \bf{Description} & Applies a binary operator to a start value and all elements of this collection, going left to right. \\
   \hline
   \bf{Signature} & \lst{def fold}$[$\lst{OV}$]$(\lst{zero}$:$~\lst{OV}, \lst{op}$:$~\lst{(OV,IV) => OV}): \lst{OV} \\
  
  \hline
  \bf{Parameters} &
      \(\begin{array}{l l}
         \lst{zero} & \text{a starting value} \\
\lst{op} & \text{the binary operator} \\
      \end{array}\) \\
       
  \hline
  
  \bf{Serialized as} & \hyperref[sec:serialization:operation:Fold]{\lst{Fold}} \\
  \hline
       
\end{tabularx}



\subsubsection{\lst{SCollection.forall} method (Code 12.6)}
\label{sec:type:SCollection:forall}
\noindent
\begin{tabularx}{\textwidth}{| l | X |}
   \hline
   \bf{Description} & Tests whether a predicate holds for all elements of this collection.
Returns \lst{true} if this collection is empty or the given predicate \lst{p}
holds for all elements of this collection, otherwise \lst{false}.
         \\
   \hline
   \bf{Signature} & \lst{def forall}(\lst{p}$:$~\lst{(IV) => Boolean}): \lst{Boolean} \\
  
  \hline
  \bf{Parameters} &
      \(\begin{array}{l l}
         \lst{p} & \text{the predicate used to test elements} \\
      \end{array}\) \\
       
  \hline
  
  \bf{Serialized as} & \hyperref[sec:serialization:operation:ForAll]{\lst{ForAll}} \\
  \hline
       
\end{tabularx}



\subsubsection{\lst{SCollection.slice} method (Code 12.7)}
\label{sec:type:SCollection:slice}
\noindent
\begin{tabularx}{\textwidth}{| l | X |}
   \hline
   \bf{Description} & Selects an interval of elements.  The returned collection is made up
  of all elements \lst{x} which satisfy the invariant:
  \lst{
     from <= indexOf(x) < until
  }
         \\
   \hline
   \bf{Signature} & \lst{def slice}(\lst{from}$:$~\lst{Int}, \lst{until}$:$~\lst{Int}): \lst{Coll[IV]} \\
  
  \hline
  \bf{Parameters} &
      \(\begin{array}{l l}
         \lst{from} & \text{the lowest index to include from this collection} \\
\lst{until} & \text{the lowest index to EXCLUDE from this collection} \\
      \end{array}\) \\
       
  \hline
  
  \bf{Serialized as} & \hyperref[sec:serialization:operation:Slice]{\lst{Slice}} \\
  \hline
       
\end{tabularx}



\subsubsection{\lst{SCollection.filter} method (Code 12.8)}
\label{sec:type:SCollection:filter}
\noindent
\begin{tabularx}{\textwidth}{| l | X |}
   \hline
   \bf{Description} & Selects all elements of this collection which satisfy a predicate.
 Returns  a new collection consisting of all elements of this collection that satisfy the given
 predicate \lst{p}. The order of the elements is preserved.
         \\
   \hline
   \bf{Signature} & \lst{def filter}(\lst{p}$:$~\lst{(IV) => Boolean}): \lst{Coll[IV]} \\
  
  \hline
  \bf{Parameters} &
      \(\begin{array}{l l}
         \lst{p} & \text{the predicate used to test elements.} \\
      \end{array}\) \\
       
  \hline
  
  \bf{Serialized as} & \hyperref[sec:serialization:operation:Filter]{\lst{Filter}} \\
  \hline
       
\end{tabularx}



\subsubsection{\lst{SCollection.append} method (Code 12.9)}
\label{sec:type:SCollection:append}
\noindent
\begin{tabularx}{\textwidth}{| l | X |}
   \hline
   \bf{Description} & Puts the elements of other collection after the elements of this collection (concatenation of 2 collections) \\
   \hline
   \bf{Signature} & \lst{def append}(\lst{other}$:$~\lst{Coll[IV]}): \lst{Coll[IV]} \\
  
  \hline
  \bf{Parameters} &
      \(\begin{array}{l l}
         \lst{other} & \text{the collection to append at the end of this} \\
      \end{array}\) \\
       
  \hline
  
  \bf{Serialized as} & \hyperref[sec:serialization:operation:Append]{\lst{Append}} \\
  \hline
       
\end{tabularx}



\subsubsection{\lst{SCollection.apply} method (Code 12.10)}
\label{sec:type:SCollection:apply}
\noindent
\begin{tabularx}{\textwidth}{| l | X |}
   \hline
   \bf{Description} & The element at given index.
 Indices start at \lst{0}; \lst{xs.apply(0)} is the first element of collection \lst{xs}.
 Note the indexing syntax \lst{xs(i)} is a shorthand for \lst{xs.apply(i)}.
 Returns the element at the given index.
 Throws an exception if \lst{i < 0} or \lst{length <= i}
         \\
   \hline
   \bf{Signature} & \lst{def apply}(\lst{i}$:$~\lst{Int}): \lst{IV} \\
  
  \hline
  \bf{Parameters} &
      \(\begin{array}{l l}
         \lst{i} & \text{the index} \\
      \end{array}\) \\
       
  \hline
  
  \bf{Serialized as} & \hyperref[sec:serialization:operation:ByIndex]{\lst{ByIndex}} \\
  \hline
       
\end{tabularx}



\subsubsection{\lst{SCollection.indices} method (Code 12.14)}
\label{sec:type:SCollection:indices}
\noindent
\begin{tabularx}{\textwidth}{| l | X |}
   \hline
   \bf{Description} & Produces the range of all indices of this collection as a new collection
 containing [0 .. length-1] values.
         \\
   \hline
   \bf{Signature} & \lst{def indices}: \lst{Coll[Int]} \\
  
  \hline
  
  \bf{Serialized as} & \hyperref[sec:serialization:operation:PropertyCall]{\lst{PropertyCall}} \\
  \hline
       
\end{tabularx}



\subsubsection{\lst{SCollection.flatMap} method (Code 12.15)}
\label{sec:type:SCollection:flatMap}
\noindent
\begin{tabularx}{\textwidth}{| l | X |}
   \hline
   \bf{Description} &  Builds a new collection by applying a function to all elements of this collection
 and using the elements of the resulting collections.
 Function \lst{f} is constrained to be of the form \lst{x => x.someProperty}, otherwise
 it is illegal.
 Returns a new collection of type \lst{Coll[B]} resulting from applying the given collection-valued function
 \lst{f} to each element of this collection and concatenating the results.
         \\
   \hline
   \bf{Signature} & \lst{def flatMap}$[$\lst{OV}$]$(\lst{f}$:$~\lst{(IV) => Coll[OV]}): \lst{Coll[OV]} \\
  
  \hline
  \bf{Parameters} &
      \(\begin{array}{l l}
         \lst{f} & \text{the function to apply to each element.} \\
      \end{array}\) \\
       
  \hline
  
  \bf{Serialized as} & \hyperref[sec:serialization:operation:MethodCall]{\lst{MethodCall}} \\
  \hline
       
\end{tabularx}



\subsubsection{\lst{SCollection.patch} method (Code 12.19)}
\label{sec:type:SCollection:patch}
\noindent
\begin{tabularx}{\textwidth}{| l | X |}
   \hline
   \bf{Description} & Produces a new collection where a slice of elements in this collection is replaced
by another collection. Returns a new collection consisting of all elements of this
collection except that \lst{replaced} elements starting from \lst{from} are
replaced by \lst{patch}. \\
   \hline
   \bf{Signature} & \footnotesize \lst{def patch}(\lst{from}$:$~\lst{Int}, \lst{patch}$:$~\lst{Coll[IV]}, \lst{replaced}$:$~\lst{Int}): \lst{Coll[IV]} \\
  
  \hline
  \bf{Parameters} &
      \(\begin{array}{l l}
         \lst{from} & \text{the index of the first replaced element} \\
\lst{patch} & \text{the replacement sequence} \\
\lst{replaced} & \text{the number of elements to drop in the original collection} \\
      \end{array}\) \\
       
  \hline
  
  \bf{Serialized as} & \hyperref[sec:serialization:operation:MethodCall]{\lst{MethodCall}} \\
  \hline
       
\end{tabularx}



\subsubsection{\lst{SCollection.updated} method (Code 12.20)}
\label{sec:type:SCollection:updated}
\noindent
\begin{tabularx}{\textwidth}{| l | X |}
   \hline
   \bf{Description} & A copy of this collection with one single replaced element.
Returns a new collection which is a copy of this collection with the element
at position \lst{index} replaced by \lst{elem}.
Throws IndexOutOfBoundsException if \lst{index} does not satisfy \lst{0 <= index < length}.
 \\
   \hline
   \bf{Signature} & \lst{def updated}(\lst{index}$:$~\lst{Int}, \lst{elem}$:$~\lst{IV}): \lst{Coll[IV]} \\
  
  \hline
  \bf{Parameters} &
      \(\begin{array}{l l}
         \lst{index} & \text{the position of the replacement} \\
\lst{elem} & \text{the replacing element} \\
      \end{array}\) \\
       
  \hline
  
  \bf{Serialized as} & \hyperref[sec:serialization:operation:MethodCall]{\lst{MethodCall}} \\
  \hline
       
\end{tabularx}



\subsubsection{\lst{SCollection.updateMany} method (Code 12.21)}
\label{sec:type:SCollection:updateMany}
\noindent
\begin{tabularx}{\textwidth}{| l | X |}
   \hline
   \bf{Description} & Returns a copy of this collection where elements at \lst{indexes} are replaced with \lst{values}. \\
   \hline
   \bf{Signature} & \footnotesize \lst{def updateMany}(\lst{indexes}$:$~\lst{Coll[Int]}, \lst{values}$:$~\lst{Coll[IV]}): \lst{Coll[IV]} \\
  
  \hline
  \bf{Parameters} &
      \(\begin{array}{l l}
         \lst{indexes} & \text{the positions of the replacement} \\
\lst{values} & \text{the values to be put in the corresponding position} \\
      \end{array}\) \\
       
  \hline
  
  \bf{Serialized as} & \hyperref[sec:serialization:operation:MethodCall]{\lst{MethodCall}} \\
  \hline
       
\end{tabularx}



\subsubsection{\lst{SCollection.indexOf} method (Code 12.26)}
\label{sec:type:SCollection:indexOf}
\noindent
\begin{tabularx}{\textwidth}{| l | X |}
   \hline
   \bf{Description} & Finds index of first occurrence of some value in this collection after or
at some start index.
Returns an index \lst{>= from} of the first element of this collection that
is equal (as determined by \lst{==}) to \lst{elem}, or \lst{-1}, if none exists.
 \\
   \hline
   \bf{Signature} & \lst{def indexOf}(\lst{elem}$:$~\lst{IV}, \lst{from}$:$~\lst{Int}): \lst{Int} \\
  
  \hline
  \bf{Parameters} &
      \(\begin{array}{l l}
         \lst{elem} & \text{the element value to search for} \\
\lst{from} & \text{the start index} \\
      \end{array}\) \\
       
  \hline
  
  \bf{Serialized as} & \hyperref[sec:serialization:operation:MethodCall]{\lst{MethodCall}} \\
  \hline
       
\end{tabularx}



\subsubsection{\lst{SCollection.zip} method (Code 12.29)}
\label{sec:type:SCollection:zip}
\noindent
\begin{tabularx}{\textwidth}{| l | X |}
   \hline
   \bf{Description} & For this collection $(x_0, \dots, x_N)$ and other collection $(y_0, \dots, y_M)$
produces a collection $((x_0, y_0), \dots, (x_K, y_K))$ where $K = min(N, M)$.
 \\
   \hline
   \bf{Signature} & \lst{def zip}$[$\lst{OV}$]$(\lst{ys}$:$~\lst{Coll[OV]}): \lst{Coll[(IV,OV)]} \\
  
  \hline
  \bf{Parameters} &
      \(\begin{array}{l l}
         \lst{ys} & \text{other collection} \\
      \end{array}\) \\
       
  \hline
  
  \bf{Serialized as} & \hyperref[sec:serialization:operation:MethodCall]{\lst{MethodCall}} \\
  \hline
       
\end{tabularx}


\subsection{Option type}
\label{sec:type:Option}

\subsubsection{\lst{SOption.isDefined} method (Code 36.2)}
\label{sec:type:SOption:isDefined}
\noindent
\begin{tabularx}{\textwidth}{| l | X |}
   \hline
   \bf{Description} & Returns \lst{true} if the option is an instance of \lst{Some}, \lst{false} otherwise. \\
   \hline
   \bf{Signature} & \lst{def isDefined}: \lst{Boolean} \\
  
  \hline
  
  \bf{Serialized as} & \hyperref[sec:serialization:operation:OptionIsDefined]{\lst{OptionIsDefined}} \\
  \hline
       
\end{tabularx}



\subsubsection{\lst{SOption.get} method (Code 36.3)}
\label{sec:type:SOption:get}
\noindent
\begin{tabularx}{\textwidth}{| l | X |}
   \hline
   \bf{Description} & Returns the option's value. The option must be nonempty. Throws exception if the option is empty. \\
   \hline
   \bf{Signature} & \lst{def get}: \lst{T} \\
  
  \hline
  
  \bf{Serialized as} & \hyperref[sec:serialization:operation:OptionGet]{\lst{OptionGet}} \\
  \hline
       
\end{tabularx}



\subsubsection{\lst{SOption.getOrElse} method (Code 36.4)}
\label{sec:type:SOption:getOrElse}
\noindent
\begin{tabularx}{\textwidth}{| l | X |}
   \hline
   \bf{Description} & Returns the option's value if the option is nonempty, otherwise
returns \lst{default}.
         \\
   \hline
   \bf{Signature} & \lst{def getOrElse}(\lst{default}$:$~\lst{T}): \lst{T} \\
  
  \hline
  \bf{Parameters} &
      \(\begin{array}{l l}
         \lst{default} & \text{the default value} \\
      \end{array}\) \\
       
  \hline
  
  \bf{Serialized as} & \hyperref[sec:serialization:operation:OptionGetOrElse]{\lst{OptionGetOrElse}} \\
  \hline
       
\end{tabularx}



\subsubsection{\lst{SOption.map} method (Code 36.7)}
\label{sec:type:SOption:map}
\noindent
\begin{tabularx}{\textwidth}{| l | X |}
   \hline
   \bf{Description} & Returns a \lst{Some} containing the result of applying \lst{f} to this option's
   value if this option is nonempty.
   Otherwise return \lst{None}.
         \\
   \hline
   \bf{Signature} & \lst{def map}$[$\lst{R}$]$(\lst{f}$:$~\lst{(T) => R}): \lst{Option[R]} \\
  
  \hline
  \bf{Parameters} &
      \(\begin{array}{l l}
         \lst{f} & \text{the function to apply} \\
      \end{array}\) \\
       
  \hline
  
  \bf{Serialized as} & \hyperref[sec:serialization:operation:MethodCall]{\lst{MethodCall}} \\
  \hline
       
\end{tabularx}



\subsubsection{\lst{SOption.filter} method (Code 36.8)}
\label{sec:type:SOption:filter}
\noindent
\begin{tabularx}{\textwidth}{| l | X |}
   \hline
   \bf{Description} & Returns this option if it is nonempty and applying the predicate \lst{p} to
  this option's value returns true. Otherwise, return \lst{None}.
         \\
   \hline
   \bf{Signature} & \lst{def filter}(\lst{p}$:$~\lst{(T) => Boolean}): \lst{Option[T]} \\
  
  \hline
  \bf{Parameters} &
      \(\begin{array}{l l}
         \lst{p} & \text{the predicate used for testing} \\
      \end{array}\) \\
       
  \hline
  
  \bf{Serialized as} & \hyperref[sec:serialization:operation:MethodCall]{\lst{MethodCall}} \\
  \hline
       
\end{tabularx}

