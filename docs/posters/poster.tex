\documentclass[article]{llncs}

\newcommand\TOCwithBibliography[2][plain]{%
  \begingroup
    \let\clearpage\relax
    \bibliographystyle{#1}
    \bibliography{#2}
  \endgroup
  \clearpage
}

\begin{document}

\title{ErgoScript: A More Expressive Alternative to Bitcoin Script}
\author{Ergo Developers}
\institute{\email{\{ergoplatform\}@protonmail.com}}

\maketitle

Every coin in Bitcoin is protected by a program in the stack-based Bitcoin Script
language. An interpreter for the language is evaluating the program against a redeeming
program~(in the same language) as well as a context (few variables information about
the state of the blockchain system). The evaluation produces a single boolean value as
a result. While Bitcoin Script allows for some contracts to be programmed, its
abilities are limited since many instructions were removed after denial-of-service and
security issues were discovered. To add new cryptographic primitives to the language
(such as ring signatures), a hard-fork is required.

Generalizing and overcoming the limitations of Bitcoin Script, we introduce a notion of
an \emph{authentication language} where a transaction verifier is running an
interpreter with three inputs: 1) a \textit{proposition} defined in terms of the
language; 2) a \textit{context} and 3) a \textit{proof}~(defined in the \emph{language
of proofs}) generated by a prover for the proposition against the same context. The
interpreter is producing a boolean value and must finish for any possible
inputs within constant time.

We designed ErgoScript as a call-by-value, higher-order functional language without
recursion with widely known concise Scala/Kotlin syntax. It supports single-assignment
blocks, tuples, optional values, indexed collections with higher-order operations,
short-cutting logicals, ternary 'if' with lazy branches. All
operations are deterministic, without side effects and all values are immutable.
ErgoScript in not turing-complete, however it is expressive enough to make the whole
transactional model of Ergo turing complete.

ErgoScript defines a guarding proposition for a coin as a logic formula which combines
predicates over a context and cryptographic statements provable via $\Sigma$-protocols
with AND, OR, k-out-of-n connectives. A user willing to spend the coin first evaluates
the proposition over known context and entire spending transaction yielding a
\emph{$\Sigma$-protocol statement}. Then the prover is turning the statement into a
signature with the help of a Fiat-Shamir transformation. A transaction verifier~(a
full-node in a blockchain setting) evaluates the proposition against the context and
checks the signature. Language expressiveness is defined by a set of predicates over
context and a set of $\Sigma$-protocol statements. We show how the latter can be
extended with a soft-fork by using versioning conventions. We propose a set of context
predicates for a Bitcoin-like cryptocurrency with a guarantee of constant verification
time. We provide several examples: zero knowledge ring and threshold signatures,
pre-issued mining rewards, crowd-funding, demurrage currency, DEX, LETS, ICO,
non-interactive CoinJoin, etc.

\TOCwithBibliography{sources}

\end{document}